\chapter{Experimentación y Resultados}
\section{Infraestructura utilizada}

\subsection{Simulación de Red}

En un principio se hicieron experimentos utilizando software 
de simulación de redes para poder lograr tamaños de 
\emph{anonymity-set} aceptables. Este software utilizado fue 
\emph{CORE}\footnote{\url{https://www.nrl.navy.mil/itd/ncs/products/core}} 
el cual permite emular distintos nodos en una red simulada 
con los parámetros que el usuario estime convenientes. Utilizando 
este software se simularon nodos conectados a través de una red local, 
donde cada uno de los nodos se comporta como un participante dentro 
del protocolo anteriormente descrito.

Además de dicho software, se levantaron instancias de 
\emph{Docker}\footnote{\url{https://www.docker.com/}}, pudiendo así 
formar una red local entre los distintos contenedores corriendo 
en una misma máquina.

Ambas simulaciones mostraron un pobre rendimiento a la hora de 
correr el protocolo, mostrando tiempos de un orden de magnitud 
mayores que los tiempos reales que se lograron \emph{a posteriori}.

\subsection{Uso de red real}

Luego de notar los pobres resultados obtenidos en las simulaciones, se 
pudo obtener acceso a equipos reales conectados a través de una red 
local. En particular se utilizó el laboratorio \emph{Lorenzo} del 
Departamento de Ciencias de la Computación de la Universidad de Chile, el 
cual cuenta con 31 computadores, cantidad razonable para poder realizar 
las pruebas correspondientes.

\section{Experimentos Realizados}

Las pruebas realizadas fueron dos:

\begin{enumerate}
	\item Tamaño de sala variable, todos los participantes enviando: se 
	varió el tamaño de la sala desde 3 hasta 30 participantes, donde en 
	cada repetición, todos los participantes presentes envían un 
	mensaje de largo 140 caracteres.
	\item Tamaño de sala fijo, algunos participantes enviando: se fijó 
	el tamaño de la sala en 30 participantes, y en cada repetición se 
	aumentaba el número de mensajes (cada uno de 140 caracteres) enviados, 
	desde 1 hasta 30.
\end{enumerate}

En cada una de las dos pruebas realizadas se midieron los siguientes 
parámetros:

\begin{itemize}
	\item Tiempos de ejecución: se midieron tres distintos tiempos, (1) tiempo 
	total de la sesión, (2) tiempo que demora en llegar el primer mensaje, y 
	(3) tiempo promedio por ronda.
	\item \emph{Profiling} de cada etapa del protocolo: se midió que porcentaje 
	del tiempo total se gasta en cada una de las etapas del protocolo, en particular 
	cuanto tiempo se ocupa en etapas de procesamiento, y cuanto tiempo se ocupa 
	en etapas de comunicación.
	\item \emph{Overhead} resultante: al finalizar cada sesión, se midió el tamaño 
	del mensaje a enviar,y se dividió por la cantidad de tiempo que 
	duró la misma sesión, obteniendo así el \emph{overhead} necesario que añade
	el protocolo para proporcionar anonimato en el envío de mensajes.
\end{itemize}

\section{Resultados}

\subsection{Tiempos de Ejecución}

\begin{figure}[h]
  \centering
    \includegraphics[scale=0.7]{logs/logs_all/times.png}
  \caption{Tiempos de Ejecución en Tamaño de sala variable}
\end{figure}

\begin{figure}[h]
  \centering
    \includegraphics[scale=0.7]{logs/logs_partial_30/times.png}
  \caption{Tiempos de Ejecución en Tamaño de sala fijo}
\end{figure}

\subsection{\emph{Profiling} de las etapas}

\begin{figure}[h]
  \centering
    \includegraphics[scale=0.3]{logs/logs_all/profile.png}
  \caption{\emph{Profiling} de etapas en Tamaño de sala variable}
\end{figure}

\begin{figure}[h]
  \centering
    \includegraphics[scale=0.3]{logs/logs_partial_30/profile.png}
  \caption{\emph{Profiling} de etapas en Tamaño de sala fijo}
\end{figure}

\subsection{\emph{Overhead} del protocolo}

\section{Discusión de los Resultados}

Las pruebas realizadas muestran varias tendencias que son dificiles de predecir 
solamente con la descripción teórica del protocolo criptográfico, mientras que otras 
son confirmaciones de resultados esperados observando netamente el diseño del 
protocolo.

Los puntos más importante a destacar son los siguientes:

\begin{itemize}
	\item Tiempos totales de ejecución:
	\item Tendencia en el aumento de tiempos:
	\item Predominancia de las etapas de comunicación:
	\item Mejora al eliminar compartición de llaves:
	\item Costo extra para asegurar anonimato:
\end{itemize}

\subsection{Escenario propuesto de uso}

Con los resultados discutidos anteriormente, se propone un escenario que logre 
porporcionar anonimato a sus participantes, y que esto no suponga pagar un alto 
costo por un rendimiento pobre de la aplicación desarrollada.

Como el protocolo es altamente sensible a (1) el tamaño de los mensajes, y (2) cantidad 
de mensajes colisionando, se propone utilizar la aplicación implementada en un contexto 
de \emph{microblogging}, con una creación de sesiones de manera periódica, otorgando 
así la posibilidad de mantener un tamaño del \emph{anonimity-set} considerable y reducir 
la posibilidad de colisiones masivas de mensajes, para así poseer un envío de mensajes 
más eficiente.

Una posible configuración del escenario anterior es el siguiente: creación de una sala 
de 30 participantes, los cuales participan cada 30 minutos en la ejecución de una sesión, 
cada uno contribuyendo para el normal desarrollo del protocolo, mientras que una fracción 
de ellos va a tener la necesidad de enviar un mensaje (de un máximo de 140 caracteres, 
\emph{á la} Twitter) en la sesión actual. Al tener una repetición periódica de la ejecución 
de sesiones, la posibilidad de una colisión masiva de mensajes en cada una de ellas es menor, 
logrando así que cada una de ellas pueda resolver de manera rápida la colisión de los mensajes 
resultantes, logrando así que cada participante que necesite enviar un mensaje, observe una 
baja latencia y pueda comunicarse de manera rápida y expedita. La ejecución de sesiones 
debe hacerse de manera constante, teniendo que en muchas de ellas no se enviará ningún mensaje, 
resultando así en una ejecución rápida y de bajo costo para todos los participantes. Además de esta 
manera se pueden agregar o expulsar a participantes entre sesiones, para así, ya sea aumentar 
el anonimato, o bien castigar a participantes que se comportaron de manera maliciosa en sesiones 
anteriores (la ejecución de las sesiones es totalmente independiente unas de otras).

El escenario anterior podría instanciarse en varios contextos donde se necesita la opción 
de comunicar mensajes de manera anónima:
\begin{itemize}
	\item Denuncias de abusos laborales dentro de una corporación privada.
	\item Foro de denunciantes (\emph{whistleblowers}) de actividades 
	realizadas por alguna institución pública.
	\item Foro de conversación entre distintos centros periodisticos (al poseer mayor 
	poder de cómputo, se puede permitir el envío de mensajes de mayor longitud).
	\item Votaciones secretas periódicas entre un grupo de interés (en vez de enviar un mensaje, 
	simplemente se envía un bit, diferenciando entre SI y NO).
\end{itemize} 