\chapter{Experimentación y Resultados}
\section{Infraestructura utilizada}

\subsection{Simulación de Red}

En un principio se hicieron experimentos utilizando software 
de simulación de redes para poder lograr tamaños de 
\emph{anonymity-set} aceptables. Este software utilizado fue 
\emph{CORE}\footnote{\url{https://www.nrl.navy.mil/itd/ncs/products/core}} 
el cual permite emular distintos nodos en una red simulada 
con los parámetros que el usuario estime convenientes. Utilizando 
este software se simularon nodos conectados a través de una red local, 
donde cada uno de los nodos se comporta como un participante dentro 
del protocolo anteriormente descrito.

Además de dicho software, se levantaron instancias de 
\emph{Docker}\footnote{\url{https://www.docker.com/}}, pudiendo así 
formar una red local entre los distintos contenedores corriendo 
en una misma máquina.

Ambas simulaciones mostraron un pobre rendimiento a la hora de 
correr el protocolo, mostrando tiempos de un orden de magnitud 
mayores que los tiempos reales que se lograron \emph{a posteriori}.

\subsection{Uso de red real}




\section{Resultados}