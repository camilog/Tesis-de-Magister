\chapter{Experimentación y Resultados}\label{cap5}

\section{Metodología Experimental}

Para poner a prueba el protocolo implementado se montaron 
distintos escenarios de experimentación, cada uno relacionados 
a distintas situaciones reales donde un sistema de mensajería 
anónima pueda ser necesario e imperante de utilizar.

La experimentación constó de dos etapas relacionadas a la 
infraestructura utilizada: (1) simulación de una red a través 
de un software especializado, y (2) utilización de una red 
real tanto para confirmar resultados entregados por la simulación, 
como para tener más precisión en los resultados entregados.

Las pruebas realizadas se centraron en la medición de tiempos reales 
de ejecución, es decir, el tiempo que un usuario del sistema 
percibe que tardó su uso en los distintos escenarios propuestos. 
Los tiempos a medir se centraron en hitos que suceden en toda ejecución 
del sistema, especificados más adelante en este capítulo.

\section{Infraestructura utilizada}

\subsection{Simulación de Red}

En un principio se hicieron experimentos utilizando software 
de simulación de redes para poder lograr tamaños de 
\emph{anonymity set} aceptables. Este software utilizado fue 
\emph{CORE}\footnote{\url{https://www.nrl.navy.mil/itd/ncs/products/core}} 
el cual permite emular distintos nodos en una red simulada 
con los parámetros que el usuario estime convenientes. Utilizando 
este software se simularon nodos conectados a través de una red local, 
donde cada uno de los nodos se comporta como un participante dentro 
del protocolo anteriormente descrito.

Además de dicho software, se levantaron instancias de 
\emph{Docker}\footnote{\url{https://www.docker.com/}}, pudiendo así 
formar una red local entre los distintos contenedores corriendo 
en una misma máquina.

Ambas simulaciones mostraron un pobre rendimiento a la hora de 
correr el protocolo, mostrando tiempos de un orden de magnitud 
mayores que los tiempos reales que se lograron \emph{a posteriori}. 
Es por ello que el autor consideró no presentar dichos resultados 
en el presente trabajo debido a la posibilidad de malinterpretarse y 
que el protocolo se viera empobrecido debido a experimentos 
mal realizados. La razón de los tiempos erróneos entregados por 
el software de simulación se centrarían principalmente en la 
inexperiencia en su uso para poder obtener como resultado tiempos 
más cercanos a los tiempos reales de ejecución. Debido a que se pudieron 
realizar pruebas en una red real, no se insistió en la mejora de los 
tiempos de la simulación, aunque queda como un punto para trabajar a 
futuro y poder probar el sistema en un sinnúmero de escenarios de 
manera más eficaz.

\subsection{Uso de red real}

Luego de notar los pobres resultados obtenidos en las simulaciones, se 
pudo obtener acceso a equipos reales conectados a través de una red 
local. En particular se utilizó el laboratorio \emph{Lorenzo} del 
Departamento de Ciencias de la Computación de la Universidad de Chile, el 
cual cuenta con 31 computadores, cantidad razonable para poder realizar 
las pruebas correspondientes.

\section{Experimentos Realizados}

Las pruebas realizadas fueron dos:

\begin{enumerate}
	\item Tamaño de sala variable, todos los participantes envían: se 
	varió el tamaño de la sala desde 3 hasta 30 participantes donde, en 
	cada repetición, todos los participantes presentes envían un 
	mensaje de 140 caracteres.
	\item Tamaño de sala fijo, algunos participantes envían: se fijó 
	el tamaño de la sala en 30 participantes, y en cada repetición se 
	aumentó el número de mensajes enviados (cada uno de 140 caracteres), 
	desde 1 hasta 30.
\end{enumerate}

En cada una de las dos pruebas realizadas se midieron los siguientes 
parámetros:

\begin{itemize}
	\item Tiempos de ejecución: se midieron tres distintos tiempos, (1) tiempo 
	total de la sesión, (2) tiempo que demora en llegar el primer mensaje, y 
	(3) tiempo promedio por ronda.
	\item \emph{Profiling} de cada etapa del protocolo: se midió que 
	porcentaje del tiempo total se gasta en cada una de las etapas del 
	protocolo, en particular cuanto tiempo se ocupa en etapas de 
	procesamiento, y cuanto tiempo se ocupa en etapas de comunicación.
	\item \emph{Overhead} resultante: al finalizar cada sesión, se midió el 
	tamaño del mensaje a enviar,y se dividió por la cantidad de tiempo que 
	duró la misma sesión, obteniendo así el \emph{overhead} necesario que añade
	el protocolo para proporcionar anonimato en el envío de mensajes.
\end{itemize}

\section{Resultados y Discusión}

\subsection{Tiempos Totales de Ejecución}

\begin{figure}[H]
  \centering
    \includegraphics[scale=0.7]{logs/logs_all/times.png}
  \caption{Tiempos de Ejecución en Tamaño de sala variable}
  \label{fig:times-variable}
\end{figure}

\begin{figure}[H]
  \centering
    \includegraphics[scale=0.7]{logs/logs_partial_30/times.png}
  \caption{Tiempos de Ejecución en Tamaño de sala fijo}
  \label{fig:times-fixed}
\end{figure}

Los tiempos totales de ejecución descritos en las Figuras 
\ref{fig:times-variable} y \ref{fig:times-fixed} 
muestran resultados alentadores con relación al tiempo necesario que hay 
que esperar para recibir mensajes de manera anónima. En el peor caso del 
experimento (30 mensajes colisionando en una sala de 30 participantes), el 
tiempo total resultó ser de 1 minuto y 30 segundos, tiempo razonable para 
leer 30 mensajes de 140 caracteres cada uno. Además, importante notar, que 
los primeros mensajes empiezan a llegar aproximadamente a los 30 segundos 
de ejecución (en el mismo caso anterior), por lo que hay una 
retro-alimentación temprana que el protocolo está funcionando y empezando a 
revelar los mensajes enviados.

Además de lo anterior, los tiempos totales muestran 
tendencias que eran esperadas. Mientras que en la Figura 
\ref{fig:times-variable} los tiempos totales siguen una tendencia 
cuadrática, esperada por el hecho que al agregar un participante más a la 
sala, se elevan al cuadrado la cantidad de conexiones necesarias (grafo 
totalmente conectado); en cambio en la Figura \ref{fig:times-fixed} 
los tiempos totales siguen una tendencia lineal, mostrando que al fijar un 
tamaño de sala (en este caso, 30 participantes), el tiempo promedio por ronda 
se mantiene constante, independiente del número de rondas necesarias para 
resolver la colisión, resultando así en una tendencia lineal a medida que 
más mensajes colisionan en la primera ronda.

\subsection{\emph{Profiling} de las etapas}

\begin{figure}[H]
  \centering
    \includegraphics[scale=0.25]{logs/logs_all/profile.png}
  \caption{\emph{Profiling} de etapas en Tamaño de sala variable}
  \label{fig:profile-variable}
\end{figure}

\begin{figure}[H]
  \centering
    \includegraphics[scale=0.25]{logs/logs_partial_30/profile.png}
  \caption{\emph{Profiling} de etapas en Tamaño de sala fijo}
  \label{fig:profile-fixed}
\end{figure}

El \emph{profiling} realizado mostrado en las Figuras 
\ref{fig:profile-variable} y \ref{fig:profile-fixed}, tanto en el caso de 
tamaño de sala variable o fijo respectivamente, muestra una 
predominancia abrumadora de las etapas que implican un alto costo en la 
capa de comunicación (recepción de \emph{commitments} y 
\emph{zero-knowledge proofs} enviados por otros participantes). 
Independiente del tamaño de la sala o el número de mensajes involucrados 
en la colisión, el tiempo de procesamiento (generación de 
\emph{commitments} y de \emph{zero-knowledge proofs}, ejecutar una ronda 
virtual o procesar la resolución de la ronda), es significativamente menor 
que las etapas de comunicación, por lo que mejoras al protocolo 
criptográfico en términos de eficiencia (hacer \emph{zero-knowledge proofs}
menos complejas, por ejemplo) no significarían un incremento muy 
importante en el tiempo total de la ejecución del protocolo, por lo que el 
trabajo futuro debería enfocarse en mejorar el rendimiento de la capa de 
comunicación, manejando de manera más eficiente el envío y recepción de 
mensajes en la sala.

Es importante mencionar una propuesta sugerida en 
el \emph{paper} original de la variante descrita en este trabajo 
\cite{franck2014dining}, la cual hace referencia a la eliminación de la etapa 
de compartición de llaves, e intercambiarla por una generación pseudo 
aleatoria de la llave, compartiendo una semilla de generación entre cada 
par de participantes. Las pruebas muestran que esta etapa se lleva entre 
un 10 y un 20 por ciento del tiempo total, por lo que su eliminación 
mostraría una mejora importante en los tiempos totales de ejecución del 
protocolo.

\subsection{\emph{Overhead} del protocolo}

\begin{figure}[H]
  \centering
    \includegraphics[scale=0.7]{logs/bandwidth.png}
  \caption{\emph{Bandwidth} del peor caso (último mensaje transmitido)}
  \label{fig:bandwidth}
\end{figure}

\subsection{Tamaños de los mensajes}

Una observación importante a realizar de la implementación desarrollada, es la 
cantidad de información (medida simplemente como el largo de los mensajes) que 
se envía en el protocolo. Se debe tomar como base el largo del mensaje 
original $m_i$ que cada participante debe comunicar. A esto se le debe agregar 
todas las \emph{zero-knowledge proofs} que necesita enviar, además de los 
\emph{commitments}, los valores necesarios que debe comunicar para establecer 
llaves compartidas con todo el resto de la sala y considerar que en realidad 
lo que se envía en el protocolo es el mensaje $M_i$. A la suma de todos esos 
valores enviados le denotaremos $S_i$. Por último hay que considerar también 
que entre más participantes colisionen sus mensajes, más veces será necesario 
reenviar esta cantidad de información, debido a que se debe desarrollar un 
mayor número de rondas reales.

Para medir este sobrecosto que agrega el protocolo, se realizaron varios 
experimentos, donde se varió el tamaño del mensaje original que desea 
comunicar cada participante y se observó cuanta información se enviaba 
finalmente (por ronda) al resto de la sala. Es importante notar que la 
cantidad de participantes presentes en la sala hace cambiar la cantidad de 
información a enviar, éste es un valor fijo (le envío a todos los 
participantes, la misma información), por lo que los resultados que se 
muestran en la Tabla \ref{table:message_sizes_table}, son por cada 
participante presente en la sala:

\begin{table}[h!]
\centering
\begin{tabular}[h!]{|c|c|c|}
\hline
$m_i$ (bytes) & $S_i$ (bytes) & $S_i / m_i$ \\ \hline
5                                   & 1633                               & 326.6       \\ \hline
10                                  & 1933                               & 193.3       \\ \hline
20                                  & 2624                               & 131.2       \\ \hline
40                                  & 3987                               & 99.7        \\ \hline
60                                  & 5288                               & 88.1        \\ \hline
80                                  & 6276                               & 78.5        \\ \hline
100                                 & 7459                               & 74.6        \\ \hline
120                                 & 8848                               & 73.7        \\ \hline
140                                 & 10281                              & 73.4        \\ \hline
200                                 & 14008                              & 70.1        \\ \hline
300                                 & 20200                              & 67.3        \\ \hline
\end{tabular}
\caption{Tamaño de mensajes enviados (por ronda)}
\label{table:message_sizes_table}
\end{table}

Cabe destacar, que entre más grande es el mensaje original a enviar, menor es 
el \emph{overhead} relativo que añade la ejecución del protocolo para asegurar 
anonimato y seguridad del sistema. Importante mencionar además que el tamaño 
del mensaje es determinado por el nodo \texttt{Directorio}, en el comienzo del 
protocolo. El nodo directorio determina un largo máximo permitido para un 
mensaje a enviar durante la sesión actual. Independiente si algún participante 
envía un mensaje de largo menor a dicho máximo, el \emph{overhead} que 
presentará es el mismo que alguien que envía un mensaje con el largo máximo 
establecido por el nodo directorio.

Por último mencionar que alcanzar el nivel de anonimato 
que ofrece el protocolo significa un sobrecosto importante que cada 
participante debe pagar, reflejado especialmente en la cantidad de 
información extra que debe enviarse con respecto al mensaje final que 
desea comunicar. Estos tamaños se ven reflejados en la Tabla 
\ref{table:message_sizes_table}, lo que muestra un altísimo sobrecosto 
relativo que experimenta cada participante (alrededor de 70 veces por 
ronda y por participante presente en la sala). Eso quiere decir que si un 
participante desea comunicar un mensaje de 140 caracteres en una sala de 
30 participantes (y tiene la ``mala suerte'' que todo el resto de la sala 
también quiere comunicar un mensaje), tendrá que enviar un total de 8.8 MB 
de información aproximadamente. Junto con esto, como es descrito en la Figura 
\ref{fig:bandwidth}, en el peor caso (es decir, 
cuando su mensaje es el último en ser revelado), experimentará un ancho de 
banda real de 1.5 bytes/sec (para comunicar su mensaje de 140 bytes, tuvo 
que esperar 90 segundos). Esto presenta un desafío muy importante a 
analizar por parte del diseño del protocolo criptográfico: ¿es posible 
alcanzar el mismo nivel de anonimato, utilizando menos cantidad de 
mensajes? ¿pudieran acortarse las rondas o utilizar 
\emph{zero-knowledge proofs} más ``cortas''? También es un desafío el 
mejorar la implementación y tratar de disminuir la cantidad de información 
enviada a través de la red, por ejemplo a través de mecanismos de 
compresión de información, pero procurando que los tiempos necesarios para 
comprimir y descomprimir la información sean tales que se evidencie una 
mejora sustancial en los tiempos totales de la ejecución del protocolo.

\subsection{Escenario propuesto de uso}

Con los resultados discutidos anteriormente, se propone un escenario que logre 
proporcionar anonimato a sus participantes, y que esto no suponga pagar un 
alto costo por un rendimiento pobre de la aplicación desarrollada.

Como el protocolo es altamente sensible a (1) el tamaño de los mensajes, y (2) 
la cantidad de mensajes involucrados en una colisión, se propone utilizar la 
aplicación implementada en un contexto de \emph{microblogging}, con una 
creación de sesiones de manera periódica, otorgando así la posibilidad de 
mantener un tamaño del \emph{anonymity set} considerable, reducir la 
posibilidad de colisiones masivas de mensajes, y proveer así un envío de 
mensajes más eficiente.

Una posible configuración del escenario anterior es el siguiente: creación de 
una sala de 30 participantes, los cuales participan cada 30 minutos en la 
ejecución de una sesión, cada uno contribuyendo para el normal desarrollo del 
protocolo, mientras que una fracción de ellos va a tener la necesidad de 
enviar un mensaje (de un máximo de 140 caracteres, \emph{á la} Twitter) en la 
sesión actual. Al tener una repetición periódica de la ejecución de sesiones, 
la posibilidad de una colisión masiva de mensajes en cada una de ellas es 
menor, logrando así que en cada una de ellas se pueda resolver de manera rápida la 
colisión de los mensajes resultantes. Efectivamente cada participante que 
necesite enviar un mensaje, puede observar una baja latencia y podrá 
comunicarse de manera rápida y expedita. La ejecución de sesiones 
debe hacerse de manera constante, teniendo en cuenta que en muchas de ellas no se 
enviará ningún mensaje, resultando así en una ejecución rápida y de bajo costo 
para todos los participantes. Además de esta manera se pueden agregar o 
expulsar a participantes entre sesiones, para así, ya sea aumentar el 
anonimato, o bien castigar a participantes que se comportaron de manera 
maliciosa en sesiones anteriores (la ejecución de las sesiones es totalmente 
independiente unas de otras).

El escenario anterior podría instanciarse en varios contextos donde se 
necesita la opción de comunicar mensajes de manera anónima:
\begin{itemize}
	\item Denuncias de abusos laborales dentro de una corporación privada.
	\item Foro de denunciantes (\emph{whistleblowers}) de actividades 
	realizadas por alguna institución pública.
	\item Foro de conversación entre distintos centros periodísticos (al 
	poseer mayor poder de cómputo, se puede permitir el envío de mensajes de 
	mayor longitud).
	\item Votaciones secretas periódicas entre un grupo de interés (en vez de 
	enviar un mensaje, 
	simplemente se envía un bit, diferenciando entre SI y NO).
\end{itemize} 

\subsection{Consideraciones sobre escenarios no experimentados}

Debido a la falta de posibilidad para usar el software de simulación, existe un 
sinnúmero de escenarios que no se pudieron experimentar y que sería interesante 
de observar como se comporta el protocolo en esos casos. En particular, 
experimentar el comportamiento del protocolo con nodos conectados a través de 
Internet (y no a través de una red local, como sí fue experimentado). Esto 
entregaría resultados más cercanos a la realidad y daría la posibilidad 
de establecer un canal anónimo para usuarios que no necesariamente comparten 
una cercanía física u organizativa, sino más bien que comparten intereses 
o simplemente que quieran aportar en fortalecer el anonimato resultante.

Otro escenario interesante de experimentar para un trabajo futuro sería 
la intermitencia en la conectividad de un subconjunto de los nodos participantes. 
Esto es que los nodos puedan conectarse/desconectarse en medio de la sesión 
que se esté ejecutando en dicho momento, y con ello observar tanto 
el comportamiento que deba tener el protocolo, como la influencia que 
esto pueda tener en los tiempos finales de ejecución. 
