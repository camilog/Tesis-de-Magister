\chapter{Diseño de la variante \emph{DC-Net}}
\section{Glosario de Términos}

\begin{itemize}
    \item Participante: agente que participa activamente del protocolo, ya sea enviando un mensaje o contribuyendo para aumentar el tamaño del \emph{anonimity-set} y así ocultar ``de mejor manera'' a los emisores de mensajes. 
    \item Ronda: serie de pasos (descritos a partir de la próxima sección) que se llevan a cabo (principalmente intercambiando mensajes entre los participantes) con el objetivo de ir reduciendo ronda a ronda los mensajes enviados por cada participante.  
    \item Mensaje: información que cada participante desea transmitir de manera anónima. Generalmente se relacionará a un conjunto de caracteres (representados como \emph{String}), en un cierto idioma, que se desea enviar.
    \item Sesión: conjunto de rondas que se necesitan realizar para que pueda ser transmitido, a lo más, un mensaje por participante. Al principio de cada sesión, se le da la oportunidad a cada participante de decidir si enviará o no un mensaje. Luego que cada participante realizó su decisión, se inicia el protocolo descrito a continuación, que tiene como objetivo que cada uno de los mensajes que se decidieron enviar, se envíen de manera anónima al resto de los participantes.
    \item Sala: conjunto de participantes corriendo una sesión del protocolo.
    \item Observador Externo: cualquier agente (a excepción de los propios participantes) que pueda estar monitoreando, tanto los mensajes resultantes del protocolo, como mensajes intermedios intercambiados entre los distintos participantes durante la realización del protocolo. A no ser que se explicite, todos los mensajes del protocolo pueden ser monitoreados por un observador externo.
    \item Adversario: agente que tiene como finalidad, ya sea encontrar al emisor de un cierto mensaje, o bien, alterar el comportamiento ``normal'' del protocolo (retrasándolo o impidiendo su total realización). Este adversario puede ser un mismo participante del protocolo (al cual llamaremos participante malicioso) o un observador externo.
\end{itemize}

\section{Diseño a primera vista}
\section{Compartición de Llaves}

Una ronda real comienza con el proceso de compartición de llaves entre cada par de participantes presentes en la sala. Este proceso debe realizarse, tomando un cuenta un posible adversario que esté monitoreando el canal de comunicación existente entre dos participantes cualquiera.

Para solucionar este problema se propone el uso del algoritmo \emph{Diffie-Hellman} \cite{diffie1976new} que permite generar un valor secreto compartido entre dos agentes, cuyo canal de comunicación es monitoreado por un adversario.

Luego que cada par de participantes $\{P_i, P_j\}$ ejecuten el algoritmo \emph{Diffie-Hellman}, obtendrán un valor secreto $k_{ij}$. Para futuros pasos del protocolo, es necesario que ejecuten este proceso dos veces, ya que es necesario que cuenten con un segundo valor compartido $r^k_{ij}$.

\subsection{\emph{Commitments} sobre las llaves}

Posteriormente cada participante $P_i$ debe realizar un \emph{commitment} a cada valor $k_{ij}$ que posee compartido con cada otro participante $P_j$, utilizando como aleatoriedad el segundo valor compartido, $r^k_{ij}$. Esto resulta en que cada participante posee $n - 1$ de los siguientes valores $c_{k_{ij}} = g^{k_{ij}} h^{r^k_{ij}}$.

Luego de esto, cada participante $P_i$ generará dos valores, el primero será la suma de todas las llaves compartidas que posee $K_i = \sum k_{ij}$. El segundo valor será un \emph{commitment} sobre dicho valor, esto es, $c_{K_i} = \prod c_{k_{ij}}$.

Finalmente cada participante $P_i$ enviará al resto de la sala vía \emph{broadcast} su \emph{commitment} $c_{K_i}$ junto con una \emph{zero-knowledge proof} demostrando que conoce los valores ``escondidos'' dentro del \emph{commitment}.

\subsection{Verificación de los valores enviados}

\todoline{Verificar zkp, multiplicacion de commitments de llaves y como encontrar al culpable}

\section{Formato del Mensaje}

Al principio de una sesión, cada participante $P_i$ debe decidir si desea comunicar un mensaje $msg$ al resto de la sala o no. De querer enviar un mensaje, habrán rondas que deberá enviar $m_i = msg$, y otras que enviará $m_i = 0$. De no querer transmitir un mensaje, en todas las rondas enviará $m_i = 0$. 

Para que el protocolo funcione correctamente, es necesario que dicho mensaje $m_i$ se concatene con otros valores, permitiendo el correcto manejo de colisiones de mensajes (es explicado más adelante en el documento). Estos valores auxiliares son los siguientes:
\begin{itemize}
    \item $b_i$: bit que indica si el participante esta, en la presente ronda, enviando un mensaje distinto a cero o no ($b_i = 1 \iff m_i \neq 0$).
    \item $pad_i$: cadena de bits aleatoria que se añade para prevenir colisión de mensajes iguales, que podría desembocar en que la sesión nunca finalice.
\end{itemize}

Con estos valores establecidos, en cada ronda cada participante $P_i$ debe formar el siguiente valor $M_i$:

\todoline{agregar diagrama de $M_i$}

\section{Correctitud del Formato del Mensaje}

Para que el protocolo tenga un correcto funcionamiento en el manejo de colisiones, es imperioso que el formato del mensaje $M_i$ se respete por todos los participantes. Para ello se debe satisfacer una de las siguientes restricciones:
\begin{itemize}
    \item Si $b_i = 0$, entonces $m_i = 0$
    \item $b_i = 1$
\end{itemize}
En la primera alternativa, se le fuerza al participante a que si esta diciendo que no envía un mensaje ($b_i = 0$), no lo envíe ($m_i = 0$). En la segunda opción, se le permite cualquier valor de $m_i$ siempre y cuando se cumpla que $b_i = 1$.

\subsection{\emph{Commitments} sobre los valores individuales}

Para poder demostrar que su mensaje $M_i$ posee un correcto formato, es necesario primero realizar \emph{commitments} a cada uno de los valores $\{m_i, pad_i, b_i\}$. Para ello, se escogen valores aleatorios $\{r_i^m, r_i^{pad}, r_i^b\}$ y se calculan los valores $c_{m_i} = g^{m_i} h^{r_i^m}; c_{pad_i} = g^{pad_i} h^{r_i^{pad}}; c_{b_i} = g^{b_i} h^{r_i^b}$ (\emph{commitments} a los valores individuales anteriormente descritos).

\subsection{\emph{Zero-knowledge Proof} sobre formato del mensaje}

Ahora es momento que cada participante genere una \emph{zero-knowledge proof} que demuestre que el mensaje $M_i$ esta bien formado. Para ello es necesario establecer primero si en la presente ronda el participante enviará un mensaje ($m_i \neq 0$) o no ($m_i = 0$).

\begin{enumerate}
    \item No envío de mensaje ($m_i = 0$): para demostrar que el formato es correcto cuando no necesito enviar un mensaje, cada participante debe demostrar la primera restricción que se mostró anteriormente, esto es, que $b_i = 0 \land m_i = 0$.
    
    Es importante notar que cuando sucede lo anterior, los valores de los \emph{commitments} anteriormente descritos quedan de la siguiente manera: $c_{m_i} = h^{r_i^m}; c_{b_i} = h^{r_i^b}$.
    
    Finalmente lo que el participante debe demostrar es que conoce los valores $\{r_i^m, r_i^b\}$ contenidos en dichos \emph{commitments}.
    \item Envío de mensaje ($m_i \neq 0$): para demostrar que el formato es correcto cuando necesito enviar un mensaje, debo demostrar la segunda restricción, esto es solamente que $b_i = 1$.
    
    Al igual que en el caso anterior el \emph{commitment} relacionado queda con un formato particular $c_{b_i} = g h^{r_i^b}$.
    
    En este caso se le va a pedir al participante demostrar que conoce $r_i^b$ en $g^{-1} c_{b_i} = h^{r_i^b}$.
\end{enumerate}

Como no se puede saber si el participante enviará o no un mensaje (comprometería su anonimato) se va a solicitar que demuestre cualquiera de las dos condiciones anteriores. La \emph{zero-knowledge proof} que necesita demostrar cada participante es la siguiente: $$\mathtt{PoK_i^{format}} = PoK\{(r_i^m, r_i^b) : (c_{m_i} = h^{r_i^m} \land c_{b_i} = h^{r_i^b}) \lor (g^{-1} c_{b_i} = h^{r_i^b})\}$$

\subsection{Envío de valores a la sala}

Después de todos los cálculos anteriores, cada participante $P_i$ debe enviar al resto de la sala vía \emph{broadcast} los siguientes valores: $\{c_{m_i}, c_{pad_i}, c_{b_i}, \mathtt{PoK_i^{format}}\}$ (los tres \emph{commitments} calculados y la demostración correspondiente).

\subsection{Verificación de la demostración}

Después de que toda la sala haya recibido todos los conjuntos de valores enviados por cada uno de los participantes, es necesario verificar que la demostración sea correcta, y para ello solamente es necesario los valores de los \emph{commitments} acompañando la \emph{zero-knowledge proof}.

\section{Demostración de conocimiento del mensaje}

Cada particiante $P_i$ debe enviar un \emph{commitment} sobre el mensaje $M_i$ construido anteriormente, con el objetivo que más adelante envíe el mensaje que ha estado construyendo durante la presente ronda. Para construir dicho \emph{commitment} utilizará los \emph{commitments} de los valores individuales enviados anteriormente. Para ello calculará $c_{M_i} = f(c_{m_i}, c_{pad_i}, c_{b_i})\footnote{\todoline{Agregar alguna descripción de la funcion f}} = g^{M_i} h^{r_i^{M}}$ (donde $r_i^{M}$ es un valor aleatorio definido por los valores $r_i^m, r_i^{pad}, r_i^b$). 

Luego de esto se necesitará que cada participante envíe una \emph{zero-knowledge proof} demostrando que conoce los valores $M_i, r_i^M$ dentro de $c_{M_i}$: $\mathtt{PoK_i^M} = PoK\{(M_i, r_i^M) : c_{M_i} = g^{M_i} h^{r_i^M}\}$.

\subsection{Verificación de la demostración}

Por cada participante $P_j$ que envía la demostración anterior, es necesario primero formar el valor $c_{M_j}$ utilizando los valores de los \emph{commitments} recibidos anteriormente de la siguiente manera: $c_{M_j} = f(c_{m_j}, c_{pad_j}, c_{b_j})$.

Luego de esto, es posible verificar la demostración $\mathtt{PoK_j^M}$ utilizando el valor $c_{M_j}$ construido anteriormente.

\section{Envío del mensaje de salida}

En este punto, cada participante $P_i$ ha demostrado la correctitud de los valores con que se ha comprometido a enviar (hasta ahora, solo ha enviado \emph{commitments} y demostraciones, ningún valor ``útil'' para la ejecución del protocolo).

Cada participante $P_i$ genererá el valor $O_i = K_i + M_i$ (denominado mensaje de salida). Este valor tiene como objetivo ``ocultar'' al mensaje $M_i$ sumándole el valor secreto $K_i$. Informalmente, cuando un participante reciba un mensaje $O_i$, no podrá discernir si este mensaje de salida contiene o no un valor $M_i \neq 0$.

Ahora bien, es necesario realizar una distinción con respecto a lo que cada participante necesita demostrar con respecto a su mensaje de salida $O_i$. Cuando se desarrolla la ronda 1, solo es necesario demostrar que dicho mensaje $O_i$ se forma a través de los valores ``ocultos'' en los \emph{commitments} $\{c_{K_i}, c_{M_i}\}$ enviados anteriormente ($c_{M_i}$ no se envió explícitamente, pero se construyó a partir de los \emph{commitments} individuales que sí se enviaron).

En las rondas posteriores, la demostración se vuelve más compleja, ya que se necesita demostrar que el valor $M_i$ que forma parte del mensaje de salida $O_i$ es, ya sea, igual a 0 (no le corresponde enviar un mensaje en dicha ronda), o bien, es el mismo mensaje que el involucrado en la colisión más próxima (más detalles serán analizados más adelante).

\subsection{Ronda 1}



\subsection{Rondas posteriores}
\section{Ronda Virtual}
\section{Resolución de la ronda}
\subsection{Primera colisión}
\subsection{Ronda sin colisión}
\subsection{Resolución de colisiones}
\section{Observación sobre tamaño de mensajes}