\chapter{Conclusiones}\label{cap7}

Las actuales alternativas que existen para proveer anonimato en comunicación 
entre personas o mientras se navega en Internet están altamente cuestionadas 
debido a nuevos ataques que vulneran la seguridad de sus usuarios. Es por ello 
que es necesario explorar nuevas alternativas y proponer nuevos protocolos 
que no posean las vulnerabilidades de las actuales soluciones y puedan ser usadas 
por distintas personas, cualesquiera sea su necesidad de anonimato.

Es en esa dirección que el presente trabajo realiza un aporte importante, avanzando 
en la investigación de crear protocolos basados en \emph{DC-Net}, protocolo que no 
posee las vulnerabilidades antes mencionadas, que sean eficientes y prácticos para 
uso masivo. 
La manera más específica para asegurar la eficiencia del protocolo es probar su 
desempeño mediante una implementación que se ajuste a todos los detalles criptográficos 
que el protocolo implica.

Finalmente la implementación realizada muestra varias conclusiones que eran difícil 
de visualizar solamente con la descripción criptográfica del protocolo:

\begin{enumerate}
    \item Escenarios ideales: las pruebas realizadas muestran que el protocolo posee una 
    alta dependencia (en términos de tiempo total del protocolo) a dos factores: 
    (1) tamaño de los mensajes, y (2) número de participantes. Por lo anterior, el escenario 
    ideal de uso del protocolo son grupos no muy grandes de participantes, enviando mensajes 
    no muy largos tampoco. Dicho escenario se ajusta muy bien a un protocolo de 
    \emph{microblogging} anónimo (un símil a \emph{Twitter} anónimo).
    \item Usabilidad del protocolo: el protocolo propuesto implica muy poca participación 
    del usuario que desea enviar un mensaje anónimo, o que simplemente desea ayudar en proveer 
    mayor anonimato para el resto de los participantes. Esto resulta en que el protocolo 
    puede volverse transparente para el usuario, y simplemente enviar mensajes tal como lo hace 
    con otras aplicaciones más populares como 
    \emph{WhatsApp}\footnote{\url{https://www.whatsapp.com/}} o 
    \emph{Telegram}\footnote{\url{https://telegram.org/}}. 
    Además de esto, el protocolo no necesita mayor infraestructura adicional (servidores externos o 
    nodos intermedios), sino 
    simplemente un nodo Directorio (que puede ser iniciado por uno de los mismos participantes), 
    que igualmente posee una aplicación móvil fácil de utilizar.
    \item Requisitos de confianza: la gran diferencia del protocolo propuesto con 
    otras propuestas realizadas en el último tiempo, es la falta del requerimiento de 
    confianza en un servidor externo al protocolo. Para poder escalar de manera importante 
    con el número de participantes, la gran mayoría de las nuevas propuestas han necesitado 
    que los participantes depositen confianza en un servidor externo ajeno a los propios 
    participantes. Si bien el protocolo descrito en este trabajo no posee la escalabilidad 
    de otras propuestas, el hecho de no obligar a los participantes a confiar en un servidor 
    externo, eleva su estándar de seguridad y privacidad, volviendo a la idea original del 
    protocolo \emph{DC-Net}, de compartir un mensaje anónimo, incluso en un contexto de 
    desconfianza hacia el resto de los participantes. 
    \item Importancia de capa de comunicación: la implementación de un protocolo criptográfico 
    implica un conocimiento, no solo de la matemática y de los conceptos criptográficos del 
    protocolo en sí, sino también de la capa de comunicación entre los distintos participantes, 
    combinando conceptos e implementación de sistemas distribuidos, junto con intercambio de 
    mensajes en redes de manera general. Muchas de las mejoras a la eficiencia de protocolos 
    criptográficos viene de la mano de las mejoras en la capa de comunicación (manejo de colas 
    de mensajes, manejo de desconexiones de usuarios, sincronización entre los distintos
     participantes, 
    etc).
    \item Construcción de aplicación final: un paso muy importante en el estudio de protocolos 
    criptográficos 
    es poder crear aplicaciones finales, permitiendo a los usuarios aprovechar los últimos avances 
    en el estudio de la criptografía. De nada sirven los avances en la teoría, sino se transforman 
    en aplicaciones finales para los usuarios, los cuales pueden mejorar su seguridad y privacidad 
    valiéndose de esos últimos avances.
\end{enumerate}


