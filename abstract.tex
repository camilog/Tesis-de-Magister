Recientes reportajes periodísticos han puesto en evidencia que el derecho a la 
privacidad en Internet ha sido fuertemente menoscabado en los últimos años 
debido a la existencia de un sinnúmero de operaciones de monitoreo realizadas 
por organismos estatales de inteligencia de distintas naciones.
Con la excusa de detectar actividades terroristas, tales organismos han 
monitoreado en forma indiscriminada y sin aparente autorización judicial las 
comunicaciones de millones de usuarios en Internet. Esto ha causado un interés 
creciente en herramientas que anonimicen las actividades de los usuarios. 
Tal es el caso del software \emph{TOR}, basado en el protocolo 
\emph{onion-routing}. Sin embargo, dicho protocolo es \emph{a priori} 
vulnerable frente a un adversario que posea un poder de monitoreo global de la 
red -- una capacidad probablemente ya existente. Es por esta razón que se hace 
necesario encontrar un nuevo protocolo para asegurar comunicaciones 
verdaderamente anónimas.

El protocolo DC-Net (\emph{Dining Cryptographers Network}), es un protocolo 
que asegura anonimato incondicional, esto es, independiente del poder de 
cómputo y la capacidad de monitoreo del adversario. Lamentablemente, el 
protocolo original presenta dos obstáculos: (1) es necesario establecer 
canales de comunicación entre todos los participantes, lo que dificulta la 
escalabilidad, y (2) esta diseñado para transmitir solo un único mensaje en 
forma simultánea, de manera que el envío de más mensajes resulta en servicio 
degradado (se dice que hay una \emph{colisión} de mensajes). Una DC-Net 
robusta debe poder manejar dichas colisiones para admitir el envío de varios 
mensajes en forma simultánea.

En este trabajo se propone una variante del protocolo DC-Net la cual emplea 
primitivas criptográficas sofisticadas (\emph{commitments} y 
\emph{zero-knowledge-proofs}) para lidiar con adversarios maliciosos y manejar 
colisiones de manera eficiente. Si bien ya existen propuestas de protocolos de 
anonimato que atacan el problema de escalabilidad, lo resuelven sacrificando 
un poco de seguridad, en particular, agregando servidores que funcionan como 
tercera parte confiable. La variante propuesta en este trabajo, si bien no 
alcanza la escalabilidad de otras propuestas, no sacrifica seguridad pues 
funciona bajo un modelo de amenaza significativamente más fuerte, uno donde el 
adversario puede monitorear todas las conexiones e incluso infiltrarse como 
participante del sistema.

A fin de evaluar esta propuesta en la práctica, se realizó una implementación 
concreta del protocolo. Esta implementación sirvió de base para la 
construcción de un prototipo de aplicación móvil, el cual efectivamente provee 
un canal anónimo entre un número potencialmente arbitrario de participantes. 
Por otro lado, la implementación realizada fue sometida a variados 
experimentos a fin de evaluar el enfoque en distintos escenarios (todos en una 
red local). Dichos experimentos muestran que el protocolo se comporta mejor en 
un escenario de foro anónimo, donde el envío de mensajes se realiza sin 
esperar una respuesta inmediata, y donde no necesariamente todos los 
participantes de dicho foro deben enviar un mensaje al mismo tiempo. 
Si bien los experimentos realizados contemplaron escenarios de hasta $30$ 
participantes, los resultados obtenidos permiten extrapolar el rendimiento del 
protocolo en situaciones más generales y con un conjunto mayor de emisores y/o 
participantes, mostrando que para escenarios de baja interactividad el enfoque 
es suficientemente práctico como para ser una alternativa efectiva de 
comunicación anónima.