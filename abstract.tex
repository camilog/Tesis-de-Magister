El derecho a la privacidad en Internet se ha visto fuertemente 
vulnerado en los últimos años, debido a revelaciones period\'isticas acerca de un 
sinnúmero de operaciones realizadas por organismos estatales de inteligencia
de distintas naciones.
Con la excusa de monitorear actividades terroristas, dichos organismos 
han monitoreado en forma indiscriminada y sin aparente autorizaci\'on judicial 
las comunicaciones de millones de usuarios en Internet. Esto ha causado un inter\'es creciente en  
herramientas que anonimicen las actividades 
de dichos usuarios. 
Tal es el caso del software TOR, basado en el 
protocolo \emph{onion-routing}. El problema actual es que 
dicho protocolo es a priori vulnerable frente a un adversario que posea 
un poder de monitoreo global de la red -- una capacidad probablemente ya existente 
por parte de los organismos de inteligencia mencionados anteriormente. Es 
por esta razón que es necesario encontrar un nuevo protocolo 
para asegurar comunicaciones verdaderamente anónimas.\\
El protocolo DC-Net (\emph{Dining Cryptographers Network}), 
es un protocolo que asegura anonimato incondicional, esto es, independiente 
del poder de cómputo y monitoreo del adversario. Los obstáculos 
que presenta el protocolo original son dos: (1) es necesario establecer 
canales de comunicación entre todos los participantes, por lo 
que no posee escalabilidad, y (2) esta diseñado para transmitir 
solo un \'unico mensaje, por lo que el envío de más mensajes resulta 
en una \emph{colisión} de mensajes. Una DC-Net debe manejar de alguna manera dichas colisiones
para admitir el envío de varios mensajes en forma simult\'anea.\\
En este trabajo se propone una variante del protocolo DC-Net, 
la cual emplea primitivas criptográficas (\emph{commitments} y 
\emph{Zero-knowledge proofs}) para lidiar con adversarios 
maliciosos y manejar colisiones de manera eficiente. 
Si bien existen propuestas de protocolos de anonimato que atacan el problema 
de escalabilidad, ellas lo resuelven sacrificando un poco de seguridad, en particular,
agregando servidores que funcionan como tercera parte confiable. 
La variante propuesta en este trabajo, si bien no alcanza la 
escalabilidad de otras propuestas, no sacrifica seguridad pues
funciona bajo un modelo de amenaza significativamente m\'as fuerte donde el
adversario puede monitorear todas las conexiones, e incluso 
infiltrarse como participante del sistema.\\
Se realizó además una implementación concreta del protocolo.
Esta implementaci\'on sirvi\'o de base para la construcción de un 
prototipo de aplicación móvil, el cual entrega un 
canal anónimo entre un número potencialmente arbitrario de participantes. 
Por otro lado, la implementación realizada fue sometida a 
variados experimentos a fin de evaluarla frente a distintos escenarios 
(todos en una red local).
Dichos experimentos muestran que el protocolo se comporta mejor 
en un escenario de foro anónimo, donde el envío de mensajes 
se realiza sin esperar una respuesta inmediata, y donde no necesariamente
todos los participantes de dicho foro requieren emitir un 
mensaje al mismo tiempo. Si bien los experimentos realizados contemplaron
escenarios de hasta 30 participantes, los resultados obtenidos permiten extrapolar los
tiempos a escenarios más generales, esto es, con un conjunto mayor 
de participantes enviando y/o participando en el protocolo. 