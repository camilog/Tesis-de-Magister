El derecho a la privacidad en Internet se ha visto fuertemente 
vulnerado en los últimos años, debido a la revelación de un 
sinnúmero de operaciones secretas realizadas por organismos de 
Estado de distintas naciones en el mundo, que con la excusa de 
monitorear actividades terroristas, se han intrometido y monitoreado 
los movimientos de millones de usuarios en Internet. Es por 
ello que el uso de herramientas que anonimicen la actividad 
que un usuario desarrolla en Internet se han masificado 
rápidamente. Tal es el caso del software TOR, basado en el 
protocolo \emph{onion-routing}. El problema actual es que 
dicho protocolo se rompre frente a un adversario que posea 
un poder de monitoreo global de la red (como se presume que 
poseen los organismos de Estado nombrados anteriormente). Es 
por esta razón que es necesario encontrar un nuevo protocolo 
para asegurar comunicaciones verdaderamente anónimas.

El protocolo \emph{DC-Net} (\emph{Dining Cryptographers Network}), 
es un protocolo que asegura anonimato incondicional (independiente 
del poder de cómputo y monitoreo del adversario). Los obstáculos 
que presenta el protocolo original son dos: (1) es necesario establecer 
canales de comunicación entre todos los participantes, por lo 
que no posee escalabilidad, y (2) esta diseñado para transmitir 
solo 1 mensaje, por lo que el envío de más mensajes resulta 
en una colisión que es necesario manejar para admitir el 
envío de más mensajes.

En este trabajo se propone una variante del protocolo \emph{DC-Net}, 
el cual emplea primitivas criptográficas (\emph{commitments} y 
\emph{Zero-knowledge proofs}) para poder lidiar con adversarios 
maliciosos y manejar la colisión de mensajes de manera eficiente. 
Si bien existen propuestas de protocolos que atacan el problema 
de escalabilidad, lo resuelven sacrificando un poco de seguridad, 
agregando servidores que funcionan como tercera parte confiable. 
La variante propuesta en este trabajo, si bien no alcanza la 
escalabilidad de otras propuestas, no sacrifica seguridad y 
funciona bajo un modelo de amenaza bastante fuerte, suponiendo 
un adversario que puede monitorear todas las conexiones, e incluso 
infiltrarse como parte de los participantes del sistema.

Además del diseño de la variante, se realizó también una implementación 
de ésta, que fue utilizada como base para la construcción de un 
prototipo de aplicación móvil, que permite mantener un 
canal anónimo entre un número determinado de participantes. 
Por otro lado, la implementación realizada fue sometida a 
variados experimentos, modificando los parámetros de los 
distintos escenarios (todos desarrollados dentro de una red local). 
Dichos experimentos muestran que el protocolo se comporta mejor 
en un escenario de foro anónimo, donde el envío de mensajes 
se realiza sin esperar una respuesta inmediata, y donde no 
todos los participantes de dicho foro requieren emitir un 
mensaje al mismo tiempo. Los experimentos realizados solo 
fueron posibles realizarlos hasta un número de 30 participantes, 
pero los resultados obtenidos permiten extrapolar dichos resultados 
para escenarios con más participantes tomando parte del grupo 
de posibles emisores de mensajes.

