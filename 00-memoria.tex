\documentclass[upright, contnum]{umemoria}
\depto{DEPARTAMENTO DE CIENCIAS DE LA COMPUTACIÓN}
\author{CAMILO JOSÉ GÓMEZ NÚÑEZ}
\title{DISEÑO E IMPLEMENTACIÓN DE UN SISTEMA DE MENSAJERÍA ANÓNIMA BASADO EN DC-NET}
\auspicio{}
\date{MARZO 2017}
\guia{ALEJANDRO HEVIA ANGULO}
\carrera{MAGÍSTER EN CIENCIAS, MENCIÓN COMPUTACIÓN}
\memoria{TESIS PARA OPTAR AL GRADO DE}
\comision{\ }{\ }{\ }

\usepackage{lipsum}

\usepackage[utf8]{inputenc}
\usepackage[T1]{fontenc}
\usepackage[nottoc,numbib]{tocbibind}
\usepackage{pstricks}
\usepackage{cancel}

\usepackage{graphicx}
\usepackage{caption}
\usepackage{subcaption}
\usepackage{float}

\usepackage{tikz}
\usetikzlibrary{shapes,arrows,patterns,mindmap,trees,positioning,fit,calc}

\usepackage{bm}

\newcommand{\nb}[3]{
		{\colorbox{#2}{\bfseries\sffamily\scriptsize\textcolor{white}{#1}}}
		{\textcolor{#2}{\sf\small$\blacktriangleright$\textit{#3}$\blacktriangleleft$}}}
		
\newcommand{\todo}[1]{\begin{center}\nb{ToDo}{red}{#1}\end{center}}
\newcommand{\todoline}[1]{\nb{ToDo}{red}{#1}}

\begin{document}

\frontmatter
\maketitle

\begin{abstract}
{\lipsum[1-4]}
\end{abstract}

\begin{dedicatoria} % opcional
Una dedicatoria corta. Por ejemplo, \emph{A los creadores de U-Campus}
\end{dedicatoria}

\begin{thanks} % opcional
\lipsum[1-2]
\end{thanks}
\cleardoublepage

\tableofcontents
% \listoftables % opcional
% \listoffigures % opcional

\mainmatter

\chapter{Introducción}
\section{Anonimato en Internet}

Desde las primeras revelaciones del grupo \emph{Wikileaks}\footnote{\url{https://wikileaks.org/}} en el año 2006, pasando por la información revelada por el ex agente de la \emph{CIA}, \emph{Edward Snowden}\footnote{\url{http://www.huffingtonpost.com/news/nsa/}}, en el año 2013, se ha generado un vuelco en la manera que los usuarios navegan por \emph{Internet}, por el hecho de tener la seguridad que todos sus movimientos están siendo registrados por organismos de Estado, a pesar de no ser amenaza a la seguridad de las naciones que suponen proteger.

Usuarios de todo el mundo reclaman por su derecho a la privacidad\footnote{\url{http://www.livescience.com/37398-right-to-privacy.
html}}, reflejándose en tomar acciones que oculten las páginas que visitan o cualquier movimiento que realicen navegando en Internet a los organismos de Estado (o cualquier tercero que ellos no autoricen) que monitorean y recopilan dichos movimientos. Existen numerosos grupos que abogan por que se haga valer el legítimo derecho a la privacidad de los usuarios de Internet, argumentando sobre la importancia del anonimato\footnote{\url{https://www.derechosdigitales.org/anonimato/}} \footnote{\url{http://www.ted.com/talks/glenn_greenwald_why_privacy_matters}}, haciendo de éste un tema altamente necesario de abordar desde una perspectiva científica, brindando herramientas y protocolos que puedan asegurar el anonimato en Internet de manera segura, eficaz y eficiente, características centrales en el desarrollo del presente trabajo.

Hoy en dia, navegar de manera anónima en Internet está vinculado fuertemente al software \emph{TOR}\footnote{\url{https://www.torproject.org/}}, el cual en su versión más popular, se traduce en un explorador web (variante del explorador \emph{Firefox}\footnote{\url{https://www.mozilla.org/en-US/firefox}}) que ejecuta el protocolo \emph{onion-routing} (detallado en la siguiente sección), el cual permite dos objetivos: (1) ocultar la identidad del usuario tanto al proveedor del servicio que está consumiendo (página web que está visitando), como también (2) acceder a servicios ``ocultos'' que solo son accesibles con el uso de \emph{TOR}. En este trabajo nos concentraremos en (1) solamente, detallando la manera en que \emph{TOR} logra ocultar la identidad del usuario del servicio. Es importante nombrar que \emph{TOR} no es la única manera de navegar de manera anónima en Internet: \emph{I2P}\footnote{\url{https://geti2p.net/en/}} es otro servicio que provee anonimato, utilizar \emph{mix-networks} (explicadas a continuación) también logra ocultar la identidad del usuario, entre otros más no detallados en este trabajo.

Anonimato, más allá del atractivo científico que tenga el tema, es una cuestión socialmente dificil de abordar, debido a que (además de los pros abordados anteriormente) posee contras que puestos en un contexto determinado, pueden llegar a generar muchas problemáticas: denuncias falsas, imposibilidad de rastrear a personas que cometen ilícitos (compra/venta de drogas, distribución de material pornográfico infantil, fraude, entre otros que el lector se pueda imaginar), realizar \emph{bullying} sin consecuencias para el agresor, etc. ¿Deberían los científicos crear y promover herramientas que permitan esas acciones? ¿Son los pros del anonimato ``tan buenos'' que justifican la creación de protocolos más seguros que proveen aun más protección a gente que pueda cometer las acciones anteriormente nombradas de manera más impoluta? Estas son preguntas que todo científico debería realizarse antes de generar un trabajo que afecta directamente la sociedad en que está envuelto. Desde el punto de vista del autor, la labor del científico es siempre generar el ``mejor conocimiento'' posible, expandir la barrera del conocimiento humano, y al mismo tiempo, alertar sobre lo que ello implica en otras esferas de la sociedad, como por ejemplo en este caso, facilitar acciones ilícitas a través de Internet. 

\section{Protocolos de Anonimato}
\subsection{\emph{mix-networks}}
\subsection{\emph{onion-routing}}
\section{La Cena de Criptógrafos}

David Chaum en el año 1985 propuso un protocolo que permitía el envío de mensajes de manera anónima entre un grupo de participantes \cite{Chaum:1985:SWI:4372.4373, chaum1988dining}. Dicho protocolo lo ejemplificó utilizando un problema llamado ``La Cena de Criptógrafos'' (\emph{Dining Cryptographers Problem}), razón del porque el protocolo subsecuente sea llamado \emph{DC-Net}.

El problema es el siguiente: están 3 criptógrafos cenando tranquilamente en una fría noche de Abril. Al terminar la cena, el mozo se acerca a su mesa y les comunica que su cena ya está pagada y que pueden irse a sus casas. Los criptógrafos, altamente consternados, se miran mutuamente y llegan a la conclusión de que pueden haber sucedido una de dos cosas: (1) uno de ellos pagó la cuenta de manera secreta (haciéndose el amable e invitando al resto sin que ellos sepan), o (2) la cuenta fue pagada por alguien distinto a ellos 3 (como la \emph{NSA} por ejemplo), revelando la intromisión del organismo de Estado en la vida de los criptógrafos. Es necesario poder dilucidar este problema, pero sin comprometer (si es que fue el caso) al criptógrafo que pagó la cuenta de manera secreta. Por lo tanto los criptógrafos necesitan saber si uno de ellos pagó la cuenta (sin saber quién) o fue alguien distinto.

David Chaum generaliza el problema de la siguiente manera: se tienen 3 participantes (los criptógrafos) donde cada uno de ellos quiere comunicar un mensaje (pagó o no pagó la cena). El resultado del protocolo debe entregar, o bien el mensaje de alguno de ellos (que pagó la cena),  sin conocer el emisor de dicho mensaje, o todos los mensajes iguales (revelando que nadie pagó la cena. concluyendo que fue un agente externo). El protocolo debe mantener el anonimato del participante que envió el mensaje tanto para el resto de los participantes como para cualquier observador externo que esté monitoreando las conversaciones.

La primera solución propuesta supone el envío de solo 1 bit de información (en el caso del problema, pagó o no pagó la cena). La solución consiste en los siguientes pasos:
\begin{enumerate}
    \item Cada par de participantes $(p_i, p_j)$ escogen un bit al azar compartido $b_{ij}$.
    \item Cada participante $p_i$ calcula $b_i$ como la operación $\oplus$\footnote{La operación $\oplus$ (\texttt{XOR}) entre dos bits, entrega 1 como resultado si y solo si ambos bits son distintos, 0 si son iguales.} entre todos los $b_{ij}$ que posea compartidos. Por ejemplo, en el caso de la cena: $b_1 = b_{12} \oplus b_{13}$, $b_2 = b_{12} \oplus b_{23}$, $b_3 = b_{13} \oplus b_{23}$.
    \item Cada participante $p_i$ define su propio $m_i$, el cual corresponderá al bit que quiere comunicar ($m_i = 1$ si pagó la cena).
    \item Cada $p_i$ revelará el valor $o_i = m_i \oplus b_i$.
    \item Cualquier $p_i$ u observador externo puede calcular el valor $D = \displaystyle\bigoplus_i o_i$.
    \item Si $D = 1$, entonces uno de los participantes envió el mensaje 1 (alguien pagó la cena), si $D = 0$, todos los participantes enviaron el mensaje 0 (la cena la pagó un agente externo a los participantes).
\end{enumerate}

\todo{Agregar diagrama con ejemplo numérico}

Luego de proponer esta solución, David Chaum realiza el análisis de seguridad del protocolo (para más detalles consultar el paper original \cite{chaum1988dining}), del cual concluye que el protocolo sugerido entrega anonimato de manera \emph{incondicional}, es decir, que incluso un adversario que posea todo el poder de cómputo posible, no puede dar con una estrategia que le permita encontrar al emisor del mensaje con mejor probabilidad que $1/n$ (donde $n$ es la cantidad total de participantes).

Finalmente Chaum propone una manera de generalizar el protocolo para utilizarlo con mensajes más largos (de más de 1 bit), lo cual será visto en la siguiente sección.

\section{\emph{DC-Net} como protocolo de anonimato}



\section{Problemas a resolver con \emph{DC-Nets}}

\subsection{Participantes Maliciosos}
\subsection{Colisión de Mensajes}

\section{Objetivos del Trabajo}
\section{Organización del Documento}
\chapter{Antecedentes}
\section{\emph{Pedersen Commitments}}

Muchas primitivas criptográficas tienen como propósito ``ocultar'' información, es decir, transformar un cierto mensaje en una secuencia que, a primera vista, parecer ser totalmente aleatoria, y que solo puede ser ``comprendida'' (o transformada en el mensaje original) con el uso de una cierta clave secreta.

Existe una variante de la descripción anterior, que se presenta cuando la clave para descifrar un cierto texto secreto, es el mismo mensaje original. A un protocolo así se le llama \emph{protocolo de commitment} (compromiso), ya que no posee como finalidad comunicar un mensaje a otra persona de manera secreta, sino que más bien comprometer a un cierto participante a un cierto mensaje, dándole la oportunidad que este mensaje pueda permanecer en secreto, pero que \emph{a posteriori} la única manera de aceptar su mensaje es que revele el valor con el cual se comprometió en un principio.

\todo{Agregar descripción matemática de un protocolo de \emph{commitment}}

Existen varias herramientas matemáticas que nos permiten instanciar un protocolo de \emph{commitment}. En particular en este trabajo se utilizó el protocolo \emph{Pedersen Commitment}\cite{pedersen1991non}, el cual se basa en la dificultad de resolver el problema del logaritmo discreto \todoline{referencia para logaritmo discreto}.

\emph{Pedersen Commitment Scheme:} Sea $G_q$ un grupo de orden $q$, en donde el problema del logaritmo discreto se crea dificil de resolver. Sean $g,h$ generadores de $G_q$ elegidos de manera aleatoria. Sea $s \in \mathbb{Z}_q$ un secreto al cual el participante se comprometerá. Además sea $r \in \mathbb{Z}_q$ elegido aleatoriamente. Se le llama un \emph{Pedersen commitment} sobre $s$ al valor: $$c := g^s h^r$$

Los \emph{Pedersen commitments} brindan dos propiedades importantes: 
  ocultamiento perfecto (\emph{unconditionally hiding}) y 
  vinculación computacional (\emph{computationally binding}). 
Esto quiere decir que, si un participante calcula un \emph{commitment} sobre un 
  cierto valor $s$, 
  (1) la persona se compromete a dicho valor (pero sin revelarlo), ya que, 
  por un lado es imposible para un tercero conocer $s$ a partir de $c$, 
  y (2) es computacionalmente difícil demostrar que dentro de $c$ existe un valor distinto a $s$.

\section{\emph{Zero-Knowledge Proofs}}

Una \emph{zero-knowledge proof} permite a una persona poder demostrar el conocimiento de cierto valor $\alpha$ que cumple una propiedad 
  (en Inglés, \emph{statement}), sin revelar el valor 
  de $\alpha$ (el testigo) en esa demostración.
Se pueden construir demostraciones para diferentes tipos de propiedades 
  (\emph{statements}), como por ejemplo: el conocimiento de un logaritmo 
  discreto; la igualdad de diferentes logaritmos con distintas bases, además de combinaciones con operadores lógicos, entre otros. 
Además existen maneras para poder demostrar propiedades genéricas sobre
  logaritmos discretos 
  \cite{camenisch1997proof}.
  
  
Existe una estrecha relación entre \emph{commitment} y \emph{zero-knowledge proofs}, ya que estás últimas generalmente demuestran propiedades que poseen valores ``escondidos'' dentro de un \emph{commitment}, por lo que se puede convencer a un tercero que un cierto valor que no deseo revelar, sigue una cierta propiedad esperada.

\section{Implementaciones de \emph{DC-Nets} relacionadas}


\chapter{Diseño de la variante \emph{DC-Net}}
\section{Glosario de Términos}

\begin{itemize}
    \item Participante: agente que participa activamente del protocolo, ya sea enviando un mensaje o contribuyendo para aumentar el tamaño del \emph{anonimity-set} y así ocultar ``de mejor manera'' a los emisores de mensajes. 
    \item Ronda: serie de pasos (descritos a partir de la próxima sección) que se llevan a cabo (principalmente intercambiando mensajes entre los participantes) con el objetivo de ir reduciendo ronda a ronda los mensajes enviados por cada participante.  
    \item Mensaje: información que cada participante desea transmitir de manera anónima. Generalmente se relacionará a un conjunto de caracteres (representados como \emph{String}), en un cierto idioma, que se desea enviar.
    \item Sesión: conjunto de rondas que se necesitan realizar para que pueda ser transmitido, a lo más, un mensaje por participante. Al principio de cada sesión, se le da la oportunidad a cada participante de decidir si enviará o no un mensaje. Luego que cada participante realizó su decisión, se inicia el protocolo descrito a continuación, que tiene como objetivo que cada uno de los mensajes que se decidieron enviar, se envíen de manera anónima al resto de los participantes.
    \item Sala: conjunto de participantes corriendo una sesión del protocolo.
    \item Observador Externo: cualquier agente (a excepción de los propios participantes) que pueda estar monitoreando, tanto los mensajes resultantes del protocolo, como mensajes intermedios intercambiados entre los distintos participantes durante la realización del protocolo. A no ser que se explicite, todos los mensajes del protocolo pueden ser monitoreados por un observador externo.
    \item Adversario: agente que tiene como finalidad, ya sea encontrar al emisor de un cierto mensaje, o bien, alterar el comportamiento ``normal'' del protocolo (retrasándolo o impidiendo su total realización). Este adversario puede ser un mismo participante del protocolo (al cual llamaremos participante malicioso) o un observador externo.
\end{itemize}

\section{Diseño a primera vista}
\section{Compartición de Llaves}

Una ronda real comienza con el proceso de compartición de llaves entre cada par de participantes presentes en la sala. Este proceso debe realizarse, tomando un cuenta un posible adversario que esté monitoreando el canal de comunicación existente entre dos participantes cualquiera.

Para solucionar este problema se propone el uso del algoritmo \emph{Diffie-Hellman} \cite{diffie1976new} que permite generar un valor secreto compartido entre dos agentes, cuyo canal de comunicación es monitoreado por un adversario.

Luego que cada par de participantes $\{P_i, P_j\}$ ejecuten el algoritmo \emph{Diffie-Hellman}, obtendrán un valor secreto $k_{ij}$. Para futuros pasos del protocolo, es necesario que ejecuten este proceso dos veces, ya que es necesario que cuenten con un segundo valor compartido $r^k_{ij}$.

\subsection{\emph{Commitments} sobre las llaves}

Posteriormente cada participante $P_i$ debe realizar un \emph{commitment} a cada valor $k_{ij}$ que posee compartido con cada otro participante $P_j$, utilizando como aleatoriedad el segundo valor compartido, $r^k_{ij}$. Esto resulta en que cada participante posee $n - 1$ de los siguientes valores $c_{k_{ij}} = g^{k_{ij}} h^{r^k_{ij}}$.

Luego de esto, cada participante $P_i$ generará dos valores, el primero será la suma de todas las llaves compartidas que posee $K_i = \sum k_{ij}$. El segundo valor será un \emph{commitment} sobre dicho valor, esto es, $c_{K_i} = \prod c_{k_{ij}}$.

\section{Correctitud del Formato del Mensaje}
\section{Demostración de conocimiento del mensaje}
\section{Envío del mensaje de salida}
\subsection{Ronda 1}
\subsection{Rondas posteriores}
\section{Ronda Virtual}
\section{Resolución de la ronda}
\subsection{Primera colisión}
\subsection{Ronda sin colisión}
\subsection{Resolución de colisiones}
\section{Observación sobre tamaño de mensajes}
\chapter{Detalles de Implementación}
\section{Tecnologías Involucradas}
\section{Arquitectura del Sistema: Nodos Directorio y Participantes}
\section{Primitivas Criptográficas}
\section{\emph{API} implementada}
\section{Aplicación Móvil}
\chapter{Experimentación y Resultados}
\section{Infraestructura utilizada}

\subsection{Simulación de Red}



\subsection{Uso de red real}




\section{Resultados}
\chapter{Trabajo Futuro}

Entre más se avanzaba en el trabajo desarrollado, mayor era la posibilidad de transformar
una investigación académica-científica en una herramienta disponible para múltiples
usuarios en el mundo que deseen transmitir información de manera anónima y simple, 
utilizando su dispositivo móvil.

Para lograr dicho objetivo aun falta trabajo que realizar, sobre todo en la capa de
comunicación entre los distintos nodos partícipes del protocolo. Las tareas que se
realizarán a futuro son las siguientes:

\begin{enumerate}
    \item Autenticar canales: la seguridad del protocolo criptográfico se basa fuertemente
    en que los canales de comunicación que existan entre los distintos nodos participantes
    sean autenticados, es decir, que cada usuario debe enviar sus mensajes firmados, para
    así el receptor de cada mensaje se asegure que el emisor es quien dice ser. Con esto
    se evitan participantes impostores, es decir, que se hagan pasar por otro 
    participante, lo que podría resultar en bajar la reputación de algún usuario,
    eliminándolo de la sala.
    
    \item Manejar desconexión de usuarios: cualquier aplicación que pretenda ser utilizada
    en ambientes ``reales'' debe manejar la posible desconexión de usuarios en cualquier 
    momento del protocolo. Actualmente esto no es manejado y simplemente la aplicación 
    deja de funcionar. Es importante además, que cualquier medida que se adopte, se 
    verifique que no influye en el anonimato y seguridad de los participantes. Por 
    ejemplo se puede adoptar la medida de ``simular'' al participante desconectado 
    suponiendo que envía mensajes vacíos, pero hay que establecer si esta medida no 
    afecta tanto la integridad del protocolo (los mensajes que aun no se reciben 
    se van a enviar satisfactoriamente) como el anonimato y seguridad de los 
    participantes que quedan involucrados, o incluso el mismo participante que sufrió 
    la desconexión (se podría saber que el participante que se desconectó envió o no 
    envió alguno de los mensajes publicados anteriormente).
    
    \item Manejar participantes maliciosos: actualmente el protocolo y la implementación 
    es capaz de encontrar a un participante malicioso, pero más allá de identificarlo 
    no emplea ninguna medida en contra de éste. Podría seguir el protocolo suponiendo 
    la desconexión del participante malicioso, lo que sería análogo al punto anterior, 
    pero tal vez existan medidas más drásticas como suspensión por un cierto tiempo 
    de participar en otras sesiones del protocolo, o derechamente la expulsión del 
    participante para siempre. 
    
    \item Optimizar recepción de mensajes: una parte importante a analizar por gente 
    más experta en el área de Redes es la manera en que se están recibiendo los 
    mensajes por parte de los participantes. Actualmente se delegó toda responsabilidad 
    a \emph{ZeroMQ}, el cual emplea una cola para no perder los mensajes entrantes 
    y la implementación actual se queda esperando entre mensajes, sin realizar 
    ninguna operación. Tal vez sea necesario revisar ese protocolo y optimizar la recepción de mensajes, abriendo múltiples \emph{sockets}, o utilizando el tiempo de espera entre mensajes para poder realizar alguna operación criptográfica pendiente, 
    y así no tener tiempo ocioso.
    
    \item Refinar protocolo criptográfico: 
    
    \item Mejorar diseño de aplicación móvil:
    
    \item Utilizar aplicación a través de Internet:
    
    \item Pruebas de seguridad:
\end{enumerate}
\chapter{Conclusión}

\todo{completar esto}

% \input{glosario.tex} % opcional

\bibliographystyle{plain}
\bibliography{bibliografia}

% \chapter{Apéndice}

\section{Apéndice A: Descripción de \emph{Zero-Knowledge Proofs}}\label{apen-a}

\subsection{Consideraciones Generales}

Las \emph{Zero-Knowledge Proofs} necesitadas para el desarrollo de este 
trabajo son del tipo \emph{proof-of-knowledge}, las cuales permiten a un 
participante, el \emph{prover}, demostrarle a otro participante, el \emph{verifier}, que conoce un cierto 
valor (o varios valores) que hacen verdadera una cierta proposición, sin la 
necesidad de revelar dicho valor. Por ejemplo, a través de esta técnica 
criptográfica una persona podría demostrar que conoce todos los 
valores que solucionan un cierto tablero de \emph{sudoku}, sin la necesidad de 
revelar estos valores.

Generalmente estas demostraciones son \emph{interactivas}, es decir, 
\emph{prover} y \emph{verifier} se comparten distintos valores durante el 
desarrollo del protocolo, lo cual finalmente convence al \emph{verifier} de la 
propiedad que se quería demostrar. Eso si, casi toda demostración interactiva puede 
transformarse a \emph{no interactiva}, es decir, sin la necesidad de 
intercambiar valores. El \emph{prover} puede demostrar el conocimiento de 
cierto valor solamente a través del cálculo de algunos valores, suponiendo que 
existen otros valores que son públicamente conocidos (en particular, conocidos 
por el \emph{verifier}). Las demostraciones utilizadas en este trabajo y 
detalladas a continuación son no interactivas.

\subsection{Logaritmo Discreto}

$$PoK\{w: y = g^w\}$$

Sean $g, y$ valores públicos, tales que $g$ es generador de un grupo cíclico $G$ de orden primo $q$, y 
$y \in G$. 
El \emph{prover} (de identidad pública 
\texttt{id}) debe demostrar que conoce el valor $w$ que hace verdadera la 
siguiente proposición: $y = g^w$. Los pasos que debe seguir son los siguientes:
\begin{enumerate}
	\item Escoger $r \in \mathbb{Z}_q$ aleatorio.
	\item Calcular $z = g^r$
	\item Calcular $b = H(z \mid\mid g \mid\mid y \mid\mid \mathtt{id})$, 
	donde $H(\cdot)$ es una función de hash resistente a colisiones (por ejemplo, 
	\texttt{SHA3}), tal que $H: \sum^{n} \rightarrow \mathbb{Z}_q$.
	\item Calcular $a = r + bw \pmod q$
	\item Enviar como demostración la tupla $\mathcal{D} = (g, z, a)$
\end{enumerate}

Finalmente, el \emph{verifier} debe recibir la demostración $\mathcal{D}$ y 
verificarla siguiendo estos pasos:
\begin{enumerate}
	\item Calcular $\hat{b} = H(z \mid\mid g \mid\mid y \mid\mid \mathtt{id})$ 
	utilizando la misma función de hash usada por el \emph{prover}.
	\item Verificar que $g^a = y^{\hat{b}} z$
\end{enumerate} 

Este protocolo también es conocido como \emph{Firma de Schnorr} y para más detalles 
sobre su seguridad y correctitud revisar \cite{schnorr1989efficient}.

\subsection{Uno de dos logaritmos discretos}

$$PoK\{x_1, x_2 : h_1 = g^{x_1} \lor h_2 = g^{x_2}\}$$

Sean $g,h_1,h_2$ valores públicos. El \emph{prover} (de identidad pública 
\texttt{id}) debe demostrar que conoce al menos uno de los valores $(x_1,x_2)$ 
que hacen verdad la siguiente proposición: $h_1 = g^{x_1} \lor h_2 = g^{x_2}$. 
Los pasos que debe seguir son los siguientes (sin pérdida de generalidad, 
supondremos que conoce el valor $x_1$):
\begin{enumerate}
	\item Escoger $c, r_1, r_2$ aleatorios.
	\item Calcular $z_1 = g^{r_1}$
	\item Calcular $z_2 = g^{r_2} h_2^{-c}$
	\item Calcular $b = H(z_1 \mid\mid z_2 \mid\mid g \mid\mid h_1 \mid\mid h_2 \mid\mid \mathtt{id})$, donde $H(\cdot)$ es una función de hash resistente a colisiones (por ejemplo, \texttt{SHA3}).
	\item Calcular $t = b - c$
	\item Calcular $a = r_1 + t x_1$
	\item Enviar como demostración la tupla $\mathcal{D} = (t, c, z_1, z_2, a, r_2)$.
\end{enumerate}

Finalmente, el \emph{verifier} debe recibir la demostración $\mathcal{D}$ y 
verificarla siguiendo estos pasos:
\begin{enumerate}
	\item Calcular $\hat{b} = H(z_1 \mid\mid z_2 \mid\mid g \mid\mid h_1 \mid\mid h_2 \mid\mid \mathtt{id})$ utilizando la misma función de hash usada por el \emph{prover}.
	\item Verificar que $\hat{b} = t + c$
	\item Verificar que $g^a = h_1^t z_1$
	\item Verificar que $g^{r_2} = h_2^c z_2$
\end{enumerate} 

El protocolo descrito anteriormente es un caso específico del esquema propuesto en \cite{cramer1994proofs}, en el cual se describe la manera de demostrar el conocimiento de un subconjunto de valores dentro de un cierto \emph{statement} (en el caso necesario para este trabajo, se demostró el conocimiento de sólo un valor de dos posibles). 

\subsection{Dos logaritmos discretos}

$$PoK\{x_1, x_2 : h_1 = g^{x_1} \land h_2 = g^{x_2}\}$$

Sean $g,h_1,h_2$ valores públicos. El \emph{prover} (de identidad pública 
\texttt{id}) debe demostrar que conoce ambos valores $(x_1,x_2)$ que hacen 
verdad la siguiente proposición: $h_1 = g^{x_1} \land h_2 = g^{x_2}$. Los 
pasos que debe seguir son los siguientes:
\begin{enumerate}
	\item Escoger $r_1, r_2$ aleatorios.
	\item Calcular $z_1 = g^{r_1} $
	\item Calcular $z_2 = g^{r_2}$
	\item Calcular $b = H(z_1 \mid\mid z_2 \mid\mid g \mid\mid h_1 \mid\mid h_2 \mid\mid \mathttt{id})$, donde $H(\cdot)$ es una función de hash resistente a colisiones (por ejemplo, \texttt{SHA3}).
	\item Calcular $a_1 = r_1 + b x_1$
	\item Calcular $a_2 = r_2 + b x_2$
	\item Enviar como demostración la tupla $\mathcal{D} = (z_1, z_2, a_1, a_2)$.
\end{enumerate}

Finalmente, el \emph{verifier} debe recibir la demostración $\mathcal{D}$ y verificarla siguiendo estos pasos:
\begin{enumerate}
	\item Calcular $\hat{b} = H(z_1 \mid\mid z_2, \mid\mid g \mid\mid h_1 \mid\mid h_2 \mid\mid \mathtt{id})$ utilizando la misma función de hash usada por el \emph{prover}.
	\item Verificar que $g^{a_1} = z_1 h_1^\hat{b}$
	\item Verificar que $g^{a_2} = z_2 h_2^\hat{b}$
\end{enumerate}

El protocolo anterior resulta de realizar en paralelo dos demostraciones como la descrita en \cite{schnorr1989efficient} (conocimiento de logaritmo discreto).

\subsection{Valores en \emph{Pedersen Commitments}}

$$PoK\{x, r : c = g^x h^r\}$$

Sean $c, g, h$ valores públicos. El \emph{prover} (de identidad pública 
\texttt{id}) debe demostrar que conoce ambos valores $(x,r)$ que hacen verdad 
la siguiente proposición: $c = g^x h^r$. Los pasos que debe seguir son los 
siguientes:
\begin{enumerate}
	\item Escoger $y, s$ aleatorios.
	\item Calcular $d = g^y h^s$
	\item Calcular $e = H(d \mid\mid g \mid\mid h \mid\mid c \mid\mid \mathtt{id})$, donde $H(\cdot)$ es una función de hash resistente a colisiones (por ejemplo, \texttt{SHA3}).
	\item Calcular $u = ex + y$
	\item Calcular $v = er + s$
	\item Enviar como demostración la tupla $\mathcal{D} = (d, u, v)$.
\end{enumerate}

Finalmente, el \emph{verifier} debe recibir la demostración $\mathcal{D}$ y 
verificarla siguiendo estos pasos:
\begin{enumerate}
	\item Calcular $\hat{e} = H(d \mid\mid g \mid\mid h \mid\mid c \mid\mid \mathttt{id})$ utilizando la misma función de hash usada por el \emph{prover}.
	\item Verificar que $g^u h^v = c^\hat{e} d$
\end{enumerate}
 % opcionales

\end{document}
