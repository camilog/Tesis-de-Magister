\chapter{Apéndice}

\section{Apéndice A: Descripción de \emph{Zero-Knowledge Proofs}}\label{apen-a}

\subsection{Consideraciones Generales}

Las \emph{Zero-Knowledge Proofs} necesitadas para el desarrollo de este 
trabajo son del tipo \emph{proof-of-knowledge}, las cuales permiten a un 
cierto \emph{prover} demostrarle a un \emph{verifier} que conoce un cierto 
valor (o varios valores) que hacen verdad una cierta proposición, sin la 
necesidad de revelar dicho valor. Por ejemplo, a través de esta técnica 
criptográfica una persona podría demostrar que conoce los todos los distintos 
valores que solucionan un cierto tablero de \emph{sudoku}, sin la necesidad de 
revelar estos valores.

Generalmente estas demostraciones son \emph{interactivas}, es decir, 
\emph{prover} y \emph{verifier} se comparten distintos valores durante el 
desarrollo del protocolo, lo cual finalmente convence al \emph{verifier} de la 
propiedad que se quería demostrar. Eso si, toda demostración interactiva puede 
transformarse a \emph{no interactiva}, es decir, sin la necesidad de 
intercambiar valores, el \emph{prover} puede demostrar el conocimiento de 
cierto valor solamente con el cálculo de algunos valores, suponiendo que 
existen otros valores que son públicamente conocidos (en particular, conocidos 
por el \emph{verifier}). Las demostraciones utilizadas en este trabajo y 
detalladas a continuación son no interactivas.

\subsection{Logaritmo Discreto}

$$PoK\{w: y = g^w\}$$

Sean $g, y$ valores públicos. El \emph{prover} (de identidad pública 
\texttt{id}) debe demostrar que conoce el valor $w$ que hace verdad la 
siguiente proposición: $y = g^w$. Los pasos que debe seguir son los siguientes:
\begin{enumerate}
	\item Escoger $r$ aleatorio.
	\item Calcular $z = g^r$
	\item Calcular $b = H(z \mid\mid g \mid\mid y \mid\mid \mathtt{id})$, 
	donde $H(\cdot)$ es una función de hash segura (por ejemplo, 
	\texttt{SHA3}).
	\item Calcular $a = r + bw$
	\item Enviar como demostración la tupla $\mathcal{D} = (g, z, a)$
\end{enumerate}

Finalmente, el \emph{verifier} debe recibir la demostración $\mathcal{D}$ y 
verificarla siguiendo estos pasos:
\begin{enumerate}
	\item Calcular $\hat{b} = H(z \mid\mid g \mid\mid y \mid\mid \mathtt{id})$ 
	utilizando la misma función de hash usada por el \emph{prover}.
	\item Verificar que $g^a = y^{\hat{b}} z$
\end{enumerate} 

Este protocolo también es conocido como \emph{Firma de Schnorr} y para más detalles 
sobre su seguridad y correctitud revisar \cite{schnorr1989efficient}.

\subsection{Uno de dos logaritmos discretos}

$$PoK\{x_1, x_2 : h_1 = g^{x_1} \lor h_2 = g^{x_2}\}$$

Sean $g,h_1,h_2$ valores públicos. El \emph{prover} (de identidad pública 
\texttt{id}) debe demostrar que conoce al menos uno de los valores $(x_1,x_2)$ 
que hacen verdad la siguiente proposición: $h_1 = g^{x_1} \lor h_2 = g^{x_2}$. 
Los pasos que debe seguir son los siguientes (sin pérdida de generalidad, 
supondremos que conoce el valor $x_1$):
\begin{enumerate}
	\item Escoger $c, r_1, r_2$ aleatorios.
	\item Calcular $z_1 = g^{r_1}$
	\item Calcular $z_2 = g^{r_2} h_2^{-c}$
	\item Calcular $b = H(z_1 \mid\mid z_2 \mid\mid g \mid\mid h_1 \mid\mid h_2 \mid\mid \mathtt{id})$, donde $H(\cdot)$ es una función de hash segura (por ejemplo, \texttt{SHA3}).
	\item Calcular $t = b - c$
	\item Calcular $a = r_1 + t x_1$
	\item Enviar como demostración la tupla $\mathcal{D} = (t, c, z_1, z_2, a, r_2)$.
\end{enumerate}

Finalmente, el \emph{verifier} debe recibir la demostración $\mathcal{D}$ y 
verificarla siguiendo estos pasos:
\begin{enumerate}
	\item Calcular $\hat{b} = H(z_1 \mid\mid z_2 \mid\mid g \mid\mid h_1 \mid\mid h_2 \mid\mid \mathtt{id})$ utilizando la misma función de hash usada por el \emph{prover}.
	\item Verificar que $\hat{b} = t + c$
	\item Verificar que $g^a = h_1^t z_1$
	\item Verificar que $g^{r_2} = h_2^c z_2$
\end{enumerate} 

El protocolo descrito anteriormente es un caso específico del esquema propuesto en \cite{cramer1994proofs}, en el cual se describe la manera de demostrar el conocimiento de un subconjunto de valores dentro de un cierto \emph{statement} (en el caso necesario para este trabajo, se demostró el conocimiento de sólo un valor de dos posibles). 

\subsection{Dos logaritmos discretos}

$$PoK\{x_1, x_2 : h_1 = g^{x_1} \land h_2 = g^{x_2}\}$$

Sean $g,h_1,h_2$ valores públicos. El \emph{prover} (de identidad pública 
\texttt{id}) debe demostrar que conoce ambos valores $(x_1,x_2)$ que hacen 
verdad la siguiente proposición: $h_1 = g^{x_1} \land h_2 = g^{x_2}$. Los 
pasos que debe seguir son los siguientes:
\begin{enumerate}
	\item Escoger $r_1, r_2$ aleatorios.
	\item Calcular $z_1 = g^{r_1} $
	\item Calcular $z_2 = g^{r_2}$
	\item Calcular $b = H(z_1 \mid\mid z_2 \mid\mid g \mid\mid h_1 \mid\mid h_2 \mid\mid \mathttt{id})$, donde $H(\cdot)$ es una función de hash segura (por ejemplo, \texttt{SHA3}).
	\item Calcular $a_1 = r_1 + b x_1$
	\item Calcular $a_2 = r_2 + b x_2$
	\item Enviar como demostración la tupla $\mathcal{D} = (z_1, z_2, a_1, a_2)$.
\end{enumerate}

Finalmente, el \emph{verifier} debe recibir la demostración $\mathcal{D}$ y verificarla siguiendo estos pasos:
\begin{enumerate}
	\item Calcular $\hat{b} = H(z_1 \mid\mid z_2, \mid\mid g \mid\mid h_1 \mid\mid h_2 \mid\mid \mathtt{id})$ utilizando la misma función de hash usada por el \emph{prover}.
	\item Verificar que $g^{a_1} = z_1 h_1^\hat{b}$
	\item Verificar que $g^{a_2} = z_2 h_2^\hat{b}$
\end{enumerate}

El protocolo anterior resulta de realizar en paralelo dos demostraciones como la descrita en \cite{schnorr1989efficient} (conocimiento de logaritmo discreto).

\subsection{Valores en \emph{Pedersen Commitments}}

$$PoK\{x, r : c = g^x h^r\}$$

Sean $c, g, h$ valores públicos. El \emph{prover} (de identidad pública 
\texttt{id}) debe demostrar que conoce ambos valores $(x,r)$ que hacen verdad 
la siguiente proposición: $c = g^x h^r$. Los pasos que debe seguir son los 
siguientes:
\begin{enumerate}
	\item Escoger $y, s$ aleatorios.
	\item Calcular $d = g^y h^s$
	\item Calcular $e = H(d \mid\mid g \mid\mid h \mid\mid c \mid\mid \mathtt{id})$, donde $H(\cdot)$ es una función de hash segura (por ejemplo, \texttt{SHA3}).
	\item Calcular $u = ex + y$
	\item Calcular $v = er + s$
	\item Enviar como demostración la tupla $\mathcal{D} = (d, u, v)$.
\end{enumerate}

Finalmente, el \emph{verifier} debe recibir la demostración $\mathcal{D}$ y 
verificarla siguiendo estos pasos:
\begin{enumerate}
	\item Calcular $\hat{e} = H(d \mid\mid g \mid\mid h \mid\mid c \mid\mid \mathttt{id})$ utilizando la misma función de hash usada por el \emph{prover}.
	\item Verificar que $g^u h^v = c^\hat{e} d$
\end{enumerate}

\section{Apéndice B: Código Fuente}\label{apen-b}

En la presente sección se muestra parte del código fuente implementado para este trabajo, 
destacando algunas secciones importantes de lo realizado para el desarrollo 
del sistema descrito anteriormente. Como fue dicho anteriormente, el código está 
programado en \emph{Java}.

\subsection{\emph{Zero-Knowledge Proof} de Valores en \emph{Pedersen Commitments}}

\begin{minted}[mathescape, breaklines=true]{java}
/**
 * Generates Proof of Knowledge that participant knows $(x, r)$ in $c = g^x h^r \pmod{p}$
 *
 * @param c commitment s.t. $c = g^x h^r \pmod{p}$
 * @param g generator of group $G_q$
 * @param x value in $\mathbb{Z}_q$
 * @param h generator of group $G_q$
 * @param r value in $\mathbb{Z}_q$
 * @param q large prime
 * @param p large prime s.t. $p = kq + 1$
 * @return ProofOfKnowledge s.t. node knows $(x,r)$ in $c = g^x h^r \pmod{p}$
 * @throws NoSuchAlgorithmException     
 * @throws UnsupportedEncodingException 
 */
public ProofOfKnowledgePedersen generateProofOfKnowledgePedersen(BigInteger c, BigInteger g, BigInteger x, BigInteger h, BigInteger r, BigInteger q, BigInteger p) throws NoSuchAlgorithmException, UnsupportedEncodingException {
    PedersenCommitment pedersenCommitment = new PedersenCommitment(g, h, q, p);

    // $y$ random value in $Z_q$
    BigInteger y = pedersenCommitment.generateRandom();
    // $s$ random value in $Z_q$ 
    BigInteger s = pedersenCommitment.generateRandom(); 

    // $d = g^y h^s \pmod{p}$
    BigInteger d = pedersenCommitment.calculateCommitment(y, s); 

    MessageDigest md = MessageDigest.getInstance("SHA-512");
    String publicValueOnHash = d.toString().concat(
            g.toString()).concat(
            h.toString()).concat(
            c.toString()).concat(
            "" + this.nodeIndex);
    md.update(publicValueOnHash.getBytes("UTF-8"));
    byte[] hashOnPublicValues = md.digest();
    // $e = H( d || g || h || c || nodeIndex ) \pmod{q}$
    BigInteger e = new BigInteger(hashOnPublicValues).mod(q); 

    // $u = ex + y$
    BigInteger u = e.multiply(x).add(y);
    // $v = er + s$ 
    BigInteger v = e.multiply(r).add(s);

    return new ProofOfKnowledgePedersen(d, u, v, nodeIndex);
}
\end{minted}

\subsection{Establecimiento de Llaves Compartidas}

\begin{minted}[mathescape, breaklines=true]{java}
/* KEY SHARING PART */
// Initialize KeyGeneration
KeyGeneration keyGeneration = new DiffieHellman(room.getRoomSize() - 1, room.getG(), room.getP(), nodeIndex, repliers, requestors, room);

// Generate Participant Node values
keyGeneration.generateParticipantNodeValues();

// Get other participants values (to produce cancellation keys)
keyGeneration.getOtherParticipantNodesValues();

// Generation of the main key round value (operation over the shared key values)
BigInteger keyRoundValue = keyGeneration.getParticipantNodeRoundKeyValue();
\end{minted}
\begin{center}
$\cdot \cdot \cdot \cdot \cdot \cdot \cdot \cdot \cdot$
\end{center}
\begin{minted}[mathescape, breaklines=true]{java}
/**
* @return other participant nodes "halves" of the shared key $g^b$
*/
@Override
public BigInteger[] getOtherParticipantNodesValues() {
    int i = 0;
    BigInteger[] otherNodesKeyHalves = new BigInteger[room.getRoomSize() - 1];
    // The "first" node doesn't have any replier sockets
    if (nodeIndex != 1)
        for (ZMQ.Socket replier : repliers) {
            // The replier wait to receive a key share
            otherNodesKeyHalves[i] = new BigInteger(replier.recvStr());
            // When the replier receives the message, replies with one of their key shares
            replier.send(participantNodeHalves[i].toString());
            i++;
        }
    // The "last" node doesn't have any requestor sockets
    if (nodeIndex != room.getRoomSize())
        for (ZMQ.Socket requestor : requestors) {
            // The requestor sends a key share
            requestor.send(participantNodeHalves[i].toString());
            // The requestor waits to receive a reply with one of the key shares
            otherNodesKeyHalves[i] = new BigInteger(requestor.recvStr());
            i++;
        }
    this.otherParticipantNodeHalves = otherNodesKeyHalves;

    i = 0;
    this.otherParticipantNodeSharedRandomValueHalves = new BigInteger[room.getRoomSize() - 1];
    // The "first" node doesn't have any replier sockets
    if (nodeIndex != 1)
        for (ZMQ.Socket replier : repliers) {
            // The replier wait to receive a key share
            this.otherParticipantNodeSharedRandomValueHalves[i] = new BigInteger(replier.recvStr());
            // When the replier receives the message, replies with one of their key shares
            replier.send(participantNodeSharedRandomValueHalves[i].toString());
            i++;
        }
    // The "last" node doesn't have any requestor sockets
    if (nodeIndex != room.getRoomSize())
        for (ZMQ.Socket requestor : requestors) {
            // The requestor sends a key share
            requestor.send(participantNodeSharedRandomValueHalves[i].toString());
            // The requestor waits to receive a reply with one of the key shares
            this.otherParticipantNodeSharedRandomValueHalves[i] = new BigInteger(requestor.recvStr());
            i++;
        }

    return otherNodesKeyHalves;
}
\end{minted}

\subsection{Recepción y Verificación de \emph{ZKP} on Messages}

\begin{minted}[mathescape, breaklines=true]{java}
/* RECEIVE COMMITMENTS AND POKs ON MESSAGES */
for (int i = 0; i < room.getRoomSize(); i++) {
    // Wait response from Receiver thread as a String (json)
    String receivedProofOfKnowledgeOnMessageJson = receiverThread.recvStr();

    // Transform String (json) to object ProofOfKnowledgePedersen
    ProofOfKnowledgePedersen receivedProofOfKnowledgeOnMessage = new Gson().fromJson(
            receivedProofOfKnowledgeOnMessageJson, ProofOfKnowledgePedersen.class);
    int receivedNodeIndex = receivedProofOfKnowledgeOnMessage.getNodeIndex();

    // Verify proof of knowledge
    if (!zkp.verifyProofOfKnowledgePedersen(receivedProofOfKnowledgeOnMessage,
            receivedCommitmentsOnMessageCurrentRound[receivedNodeIndex - 1],
            room.getG(), room.getH(), room.getQ(), room.getP()))
        System.err.println("WRONG PoK on Message. Round: " + currentRound + ", Node: " +
                receivedProofOfKnowledgeOnMessage.getNodeIndex());
}
\end{minted}

\subsection{Conexiones al Nodo \texttt{Directorio}}

\begin{minted}[mathescape, breaklines=true]{java}
// Create the PUB socket and bind it to the port 5555
ZMQ.Socket publisher = context.createSocket(ZMQ.PUB);
publisher.bind("tcp://*:6555");

// Create the PULL socket and bind it to the port 5554
ZMQ.Socket pull = context.createSocket(ZMQ.PULL);
pull.bind("tcp://*:6554");

ZMQ.Socket[] participantsConnectedPush = new ZMQ.Socket[roomSize];
// Wait to receive <numberOfNodes> connections from each node that wants to send a message in this room
for (int i = 0; i < roomSize; i++) {
    // Receive a message from the PULL socket, which corresponds to the IP address of this node
    String messageReceived = pull.recvStr();
    // Assign an index to this node and store it in the nodesInTheRoom with his correspondent IP address
    this.infoFromDirectory.nodes[i] = new ParticipantNodeInfoFromDirectory(i+1, messageReceived);
    // Send ACK to participant that just connected
    participantsConnectedPush[i] = context.createSocket(ZMQ.PUSH);
    participantsConnectedPush[i].connect("tcp://" + messageReceived + ":5554");
    participantsConnectedPush[i].send("ack");
    // Notify of change to observer
    participantConnected.setValue(messageReceived);
    // Send message to all connected participants of how many participants left to complete the room
    sendHowManyParticipantsLeft(participantsConnectedPush, i+1);
}
\end{minted}
