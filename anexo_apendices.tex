\chapter{Apéndice}

\section{Apéndice A: Descripción de \emph{Zero-Knowledge Proofs}}\label{apen-a}

\subsection{Consideraciones Generales}

Las \emph{Zero-Knowledge Proofs} necesitadas para el desarrollo de este 
trabajo son del tipo \emph{proof-of-knowledge}, las cuales permiten a un 
participante, el \emph{prover}, demostrarle a otro participante, el \emph{verifier}, que conoce un cierto 
valor (o varios valores) que hacen verdadera una cierta proposición, sin la 
necesidad de revelar dicho valor. Por ejemplo, a través de esta técnica 
criptográfica una persona podría demostrar que conoce todos los 
valores que solucionan un cierto tablero de \emph{sudoku}, sin la necesidad de 
revelar estos valores.

Generalmente estas demostraciones son \emph{interactivas}, es decir, 
\emph{prover} y \emph{verifier} se comparten distintos valores durante el 
desarrollo del protocolo, lo cual finalmente convence al \emph{verifier} de la 
propiedad que se quería demostrar. Eso si, casi toda demostración interactiva puede 
transformarse a \emph{no interactiva}, es decir, sin la necesidad de 
intercambiar valores. El \emph{prover} puede demostrar el conocimiento de 
cierto valor solamente a través del cálculo de algunos valores, suponiendo que 
existen otros valores que son públicamente conocidos (en particular, conocidos 
por el \emph{verifier}). Las demostraciones utilizadas en este trabajo y 
detalladas a continuación son no interactivas.

\subsection{Logaritmo Discreto}

$$PoK\{w: y = g^w\}$$

Sean $g, y$ valores públicos, tales que $g$ es generador de un grupo cíclico $G$ de orden primo $q$, y 
$y \in G$. 
El \emph{prover} (de identidad pública 
\texttt{id}) debe demostrar que conoce el valor $w$ que hace verdadera la 
siguiente proposición: $y = g^w$. Los pasos que debe seguir son los siguientes:
\begin{enumerate}
	\item Escoger $r \in \mathbb{Z}_q$ aleatorio.
	\item Calcular $z = g^r$
	\item Calcular $b = H(z \mid\mid g \mid\mid y \mid\mid \mathtt{id})$, 
	donde $H(\cdot)$ es una función de hash resistente a colisiones (por ejemplo, 
	\texttt{SHA3}), tal que $H: \sum^{n} \rightarrow \mathbb{Z}_q$.
	\item Calcular $a = r + bw \pmod q$
	\item Enviar como demostración la tupla $\mathcal{D} = (g, z, a)$
\end{enumerate}

Finalmente, el \emph{verifier} debe recibir la demostración $\mathcal{D}$ y 
verificarla siguiendo estos pasos:
\begin{enumerate}
	\item Calcular $\hat{b} = H(z \mid\mid g \mid\mid y \mid\mid \mathtt{id})$ 
	utilizando la misma función de hash usada por el \emph{prover}.
	\item Verificar que $g^a = y^{\hat{b}} z$
\end{enumerate} 

Este protocolo también es conocido como \emph{Firma de Schnorr} y para más detalles 
sobre su seguridad y correctitud revisar \cite{schnorr1989efficient}.

\subsection{Uno de dos logaritmos discretos}

$$PoK\{x_1, x_2 : h_1 = g^{x_1} \lor h_2 = g^{x_2}\}$$

Sean $g,h_1,h_2$ valores públicos. El \emph{prover} (de identidad pública 
\texttt{id}) debe demostrar que conoce al menos uno de los valores $(x_1,x_2)$ 
que hacen verdad la siguiente proposición: $h_1 = g^{x_1} \lor h_2 = g^{x_2}$. 
Los pasos que debe seguir son los siguientes (sin pérdida de generalidad, 
supondremos que conoce el valor $x_1$):
\begin{enumerate}
	\item Escoger $c, r_1, r_2$ aleatorios.
	\item Calcular $z_1 = g^{r_1}$
	\item Calcular $z_2 = g^{r_2} h_2^{-c}$
	\item Calcular $b = H(z_1 \mid\mid z_2 \mid\mid g \mid\mid h_1 \mid\mid h_2 \mid\mid \mathtt{id})$, donde $H(\cdot)$ es una función de hash resistente a colisiones (por ejemplo, \texttt{SHA3}).
	\item Calcular $t = b - c$
	\item Calcular $a = r_1 + t x_1$
	\item Enviar como demostración la tupla $\mathcal{D} = (t, c, z_1, z_2, a, r_2)$.
\end{enumerate}

Finalmente, el \emph{verifier} debe recibir la demostración $\mathcal{D}$ y 
verificarla siguiendo estos pasos:
\begin{enumerate}
	\item Calcular $\hat{b} = H(z_1 \mid\mid z_2 \mid\mid g \mid\mid h_1 \mid\mid h_2 \mid\mid \mathtt{id})$ utilizando la misma función de hash usada por el \emph{prover}.
	\item Verificar que $\hat{b} = t + c$
	\item Verificar que $g^a = h_1^t z_1$
	\item Verificar que $g^{r_2} = h_2^c z_2$
\end{enumerate} 

El protocolo descrito anteriormente es un caso específico del esquema propuesto en \cite{cramer1994proofs}, en el cual se describe la manera de demostrar el conocimiento de un subconjunto de valores dentro de un cierto \emph{statement} (en el caso necesario para este trabajo, se demostró el conocimiento de sólo un valor de dos posibles). 

\subsection{Dos logaritmos discretos}

$$PoK\{x_1, x_2 : h_1 = g^{x_1} \land h_2 = g^{x_2}\}$$

Sean $g,h_1,h_2$ valores públicos. El \emph{prover} (de identidad pública 
\texttt{id}) debe demostrar que conoce ambos valores $(x_1,x_2)$ que hacen 
verdad la siguiente proposición: $h_1 = g^{x_1} \land h_2 = g^{x_2}$. Los 
pasos que debe seguir son los siguientes:
\begin{enumerate}
	\item Escoger $r_1, r_2$ aleatorios.
	\item Calcular $z_1 = g^{r_1} $
	\item Calcular $z_2 = g^{r_2}$
	\item Calcular $b = H(z_1 \mid\mid z_2 \mid\mid g \mid\mid h_1 \mid\mid h_2 \mid\mid \mathttt{id})$, donde $H(\cdot)$ es una función de hash resistente a colisiones (por ejemplo, \texttt{SHA3}).
	\item Calcular $a_1 = r_1 + b x_1$
	\item Calcular $a_2 = r_2 + b x_2$
	\item Enviar como demostración la tupla $\mathcal{D} = (z_1, z_2, a_1, a_2)$.
\end{enumerate}

Finalmente, el \emph{verifier} debe recibir la demostración $\mathcal{D}$ y verificarla siguiendo estos pasos:
\begin{enumerate}
	\item Calcular $\hat{b} = H(z_1 \mid\mid z_2, \mid\mid g \mid\mid h_1 \mid\mid h_2 \mid\mid \mathtt{id})$ utilizando la misma función de hash usada por el \emph{prover}.
	\item Verificar que $g^{a_1} = z_1 h_1^\hat{b}$
	\item Verificar que $g^{a_2} = z_2 h_2^\hat{b}$
\end{enumerate}

El protocolo anterior resulta de realizar en paralelo dos demostraciones como la descrita en \cite{schnorr1989efficient} (conocimiento de logaritmo discreto).

\subsection{Valores en \emph{Pedersen Commitments}}

$$PoK\{x, r : c = g^x h^r\}$$

Sean $c, g, h$ valores públicos. El \emph{prover} (de identidad pública 
\texttt{id}) debe demostrar que conoce ambos valores $(x,r)$ que hacen verdad 
la siguiente proposición: $c = g^x h^r$. Los pasos que debe seguir son los 
siguientes:
\begin{enumerate}
	\item Escoger $y, s$ aleatorios.
	\item Calcular $d = g^y h^s$
	\item Calcular $e = H(d \mid\mid g \mid\mid h \mid\mid c \mid\mid \mathtt{id})$, donde $H(\cdot)$ es una función de hash resistente a colisiones (por ejemplo, \texttt{SHA3}).
	\item Calcular $u = ex + y$
	\item Calcular $v = er + s$
	\item Enviar como demostración la tupla $\mathcal{D} = (d, u, v)$.
\end{enumerate}

Finalmente, el \emph{verifier} debe recibir la demostración $\mathcal{D}$ y 
verificarla siguiendo estos pasos:
\begin{enumerate}
	\item Calcular $\hat{e} = H(d \mid\mid g \mid\mid h \mid\mid c \mid\mid \mathttt{id})$ utilizando la misma función de hash usada por el \emph{prover}.
	\item Verificar que $g^u h^v = c^\hat{e} d$
\end{enumerate}
