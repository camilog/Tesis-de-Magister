\chapter{Detalles de Implementación}

La parte esencial de este trabajo esta dado por llevar a cabo una implementación que refleje el diseño propuesto en el capitulo anterior, brindando así una herramienta real a la comunidad que permita la comunicación anónima entre distintas partes interesadas.

Como fue explicado anteriormente, no existen muchas implementaciones de protocolos basados en DC-Net, por lo que llevar a cabo este trabajo conllevó soslayar muchos obstáculos que se detallan a continuación.

\section{Tecnologías Involucradas}

\begin{enumerate}
    \item \underline{Lenguaje de Programación:} para llevar a cabo este proyecto se escogió realizarlo en \emph{Java}, debido principalmente a que un objetivo era poder realizar una aplicación móvil como prueba de concepto de lo implementado, y el sistema operativo móvil más común hoy en día es \emph{Android}\todoline{agregar pie de pagina que apoye esto}, el cual está basado en \emph{Java}. No existe una razón de fondo (mayor \emph{performance} o expresividad del lenguaje), por el que se escogió \emph{Java}, por lo que los mismos resultados se pueden lograr si se utiliza otro lenguaje más común en otras implementaciones de DC-Net, como sería \emph{C++}.
    \item \underline{Capa de comunicación:} para la conexión entre las distintas partes se utilizó \emph{ZeroMQ}, el cual es un \emph{framework} de concurrencia, que permite desligarse de lidiar con problemas típicos de mensajería distribuida (desconexiones, pérdida de datos, etc.) y concentrarse únicamente en la lógica de la aplicación. En palabras de sus autores, "\emph{ZeroMQ} son sockets con esteroides". Con \emph{ZeroMQ} se simplifica la implementación de \emph{broadcasting} o conexiones punto a punto entre los distintos participantes (ambos tipos de conexiones necesarias para el protocolo).
\end{enumerate}

\section{Arquitectura del Sistema: Nodos Directorio y Participantes}

Un desafío importante a considerar en la implementación (y que el diseño del protocolo no se preocupa) es como descubrir la locación del resto de los participantes que formarán parte del \emph{anonymity-set}. Para ello se tomó la decisión de contar, además de los nodos participantes, con un nodo Directorio. Este nodo funcionará como punto de entrada al \emph{anonymity-set} y será el responsable de informar la dirección IP de cada uno de los nodos participantes. Además de esto, el nodo Directorio tiene como responsabilidad establecer los parámetros necesarios para correr el protocolo (número de participantes, valores públicos para realizar los \emph{commitments}, entre otros), por lo que también se vuelve un punto de control dentro del protocolo. Un aspecto importante es que todo lo que relaciona el protocolo con mantener anonimato (envío y verificaciones de demostraciones de seguridad) no pasa por el nodo Directorio, por lo que su incorporación no altera en nada el requisito de seguridad buscado (emisores de mensajes anónimos).

Esta tarea también se puede realizar entre los propios nodos participantes, sin la necesidad de incorporar un nodo Directorio. Si bien en esta investigación se priorizó la facilidad de implementar la variante utilizando el nodo adicional, correr alguna variante de \emph{gossip protocol} para informar la identidad de participantes nuevos que vayan entrando a la sala podría solucionar el problema. Esta segunda variante además tiene la ventaja de no poseer un punto vulnerable (que sería el nodo Directorio), evitando ataques directos al nodo Directorio, retrasando (o incluso imposibilitando) la creación del \emph{anonimity-set}. Reiterar que se implementó el nodo Directorio por simplicidad, pero se tienen en cuenta los posibles ataques que puede recibir el sistema, que no tienen incidencia en romper el anonimato que brinda el protocolo.

Actualmente el nodo Directorio inicializa estableciendo los parámetros públicos del protocolo y publica su dirección IP. Luego, cada nodo Participante que se quiera unir se conecta a la dirección pública del Directorio y espera que se complete la cuota de participantes establecida en un comienzo. Cuando se conectan $n$ participantes al Directorio, éste informa la dirección IP de cada uno de los participantes a todo el resto, para que posteriormente inicien el protocolo solo enviándose mensajes entre ellos, finalizando así la labor del Directorio.

\section{Primitivas Criptográficas}

En la implementación actual, la gran mayoría de las primitivas criptográficas han sido implementadas desde cero, valiéndose principalmente de la biblioteca para manejar números grandes de \emph{Java}, \emph{BigInteger}\footnote{\url{https://docs.oracle.com/javase/7/docs/api/java/math/BigInteger.html}}. Si bien esto no es una práctica recomendada (lo ideal es utilizar bibliotecas criptográficas ya probadas por la comunidad), no se ha descubierto ninguna que se adecúe a las necesidades requeridas por el protocolo (\emph{Pedersen Commitments} y \emph{ZKP} asociadas). Este punto es algo importante a resolver ya que, como fue dicho, la implementación de primitivas criptográficas no es recomendado y es imperante utilizar implementaciones ya probadas y verificadas por la comunidad, como lo podría ser Charm-Crypto\footnote{\url{http://charm-crypto.com/index.html}} o Scapi\footnote{\url{https://scapi.readthedocs.io/en/latest/}}. De todas maneras, la criptografía implementada se desarrolló utilizando interfaces, por lo que cuando se descubra una librería que cumpla los requerimientos criptográficos que se buscan, su implantación sea realizada de manera fácil.

\section{\emph{API} implementada}

Como fue dicho anteriormente, una contribución importante de este trabajo sería el hecho de estar apuntado a implementar una \emph{API} disponible para toda la comunidad que quiera desarrollar una aplicación que esté basada en el protocolo anteriormente descrito. 

La implementación anterior fue empaquetada como librería \emph{Java}, utilizable por otras aplicaciones al importar el archivo \emph{.jar} generado. Esta librería quedó con los siguientes métodos de manera pública, formando así la \emph{API} disponible:

\begin{itemize}
    \item \texttt{DCNETProtocol class:} clase base que entrega la \emph{API} que es necesario instanciar para utilizar los métodos descritos a continuación. 
    \item \texttt{boolean connectToDirectory():}
    \item \texttt{void setMessageToSend(String s, boolean b):}
    \item \texttt{void runProtocol(PrintStream p):}
    \item \texttt{ObservableParticipantsLeft:}
    \item \texttt{ObservableMessagesArrived:}
\end{itemize}

\section{Aplicación Móvil}

Para poder probar el uso de la \emph{API} implementada, se desarrolló una aplicación móvil prototipo, que tiene como objetivo simplificar 

\section{Observación sobre tamaños de los mensajes}

Una observación importante a realizar de la implementación realizada, es la cantidad de información (medida simplemente como el largo de los mensajes) que se envía en el protocolo. Se debe tomar como base el largo del mensaje original $m_i$ que cada participante debe comunicar. A esto se le debe agregar todas las \emph{zero-knowledge proofs} que necesita enviar, además de los \emph{commitments}, los valores necesarios que debe comunicar para establecer llaves compartidas con todo el resto de la sala y considerar que en realidad lo que se envía en el protocolo es el mensaje $M_i$ (a la suma de todos esos valores enviados le denotaremos $S_i$). Por último hay que considerar también que entre más participantes colisionen sus mensajes, más veces será necesario reenviar esta cantidad de información, debido a que se debe desarrollar un mayor número de rondas reales.

Para medir este sobrecosto que agrega el protocolo, se realizaron varios experimentos, donde se varió el tamaño del mensaje original que desea comunicar cada participante y se observó cuanta información se enviaba finalmente (por ronda) al resto de la sala. Importante notar que, si bien el tamaño de la sala también hace variar la cantidad de información enviada, lo hace en un factor muy menor (con 256 participantes, se agrega 1 byte), por lo que esa variable se dejó de lado. Los resultados se muestran en la Tabla \ref{table:message_sizes_table}.

\begin{table}[h!]
\centering
\begin{tabular}[h!]{|c|c|c|}
\hline
$m_i$ (bytes) & $S_i$ (bytes) & $S_i / m_i$ \\ \hline
5                                   & 1633                               & 326.6       \\ \hline
10                                  & 1933                               & 193.3       \\ \hline
20                                  & 2624                               & 131.2       \\ \hline
40                                  & 3987                               & 99.7        \\ \hline
60                                  & 5288                               & 88.1        \\ \hline
80                                  & 6276                               & 78.5        \\ \hline
100                                 & 7459                               & 74.6        \\ \hline
120                                 & 8848                               & 73.7        \\ \hline
140                                 & 10281                              & 73.4        \\ \hline
200                                 & 14008                              & 70.1        \\ \hline
300                                 & 20200                              & 67.3        \\ \hline
\end{tabular}
\caption{Tamaño de mensajes enviados (por ronda)}
\label{table:message_sizes_table}
\end{table}

Cabe destacar, que entre más grande es el mensaje original a enviar, menor es el \emph{overhead} relativo que añade la ejecución del protocolo para asegurar anonimato y seguridad del sistema. Importante mencionar además que el tamaño del mensaje es determinado por el nodo directorio, en el comienzo del protocolo. El directorio determina un largo máximo permitido para un mensaje. Independiente si algún participa envía un mensaje de largo menor a dicho máximo, el \emph{overhead} que presentará es el mismo que alguien que envía un mensaje con el largo máximo establecido por el nodo directorio.