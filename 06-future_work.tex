\chapter{Trabajo Futuro}

Entre más se avanzaba en el trabajo desarrollado, mayor era la posibilidad de transformar
una investigación académica-científica en una herramienta disponible para múltiples
usuarios en el mundo que deseen transmitir información de manera anónima y simple, 
utilizando su dispositivo móvil.

Para lograr dicho objetivo aun falta trabajo que realizar, sobre todo en la capa de
comunicación entre los distintos nodos partícipes del protocolo. Las tareas que se
realizarán a futuro son las siguientes:

\begin{enumerate}
    \item Autenticar canales: la seguridad del protocolo criptográfico se basa fuertemente
    en que los canales de comunicación que existan entre los distintos nodos participantes
    sean autenticados, es decir, que cada usuario debe enviar sus mensajes firmados, para
    así el receptor de cada mensaje se asegure que el emisor es quien dice ser. Con esto
    se evitan participantes impostores, es decir, que se hagan pasar por otro 
    participante, lo que podría resultar en bajar la reputación de algún usuario,
    eliminándolo de la sala.
    
    \item Manejar desconexión de usuarios: cualquier aplicación que pretenda ser utilizada
    en ambientes ``reales'' debe manejar la posible desconexión de usuarios en cualquier 
    momento del protocolo. Actualmente esto no es manejado y simplemente la aplicación 
    deja de funcionar. Es importante además, que cualquier medida que se adopte, se 
    verifique que no influye en el anonimato y seguridad de los participantes. Por 
    ejemplo se puede adoptar la medida de ``simular'' al participante desconectado 
    suponiendo que envía mensajes vacíos, pero hay que establecer si esta medida no 
    afecta tanto la integridad del protocolo (los mensajes que aun no se reciben 
    se van a enviar satisfactoriamente) como el anonimato y seguridad de los 
    participantes que quedan involucrados, o incluso el mismo participante que sufrió 
    la desconexión (se podría saber que el participante que se desconectó envió o no 
    envió alguno de los mensajes publicados anteriormente).
    
    \item Manejar participantes maliciosos: actualmente el protocolo y la implementación 
    es capaz de encontrar a un participante malicioso, pero más allá de identificarlo 
    no emplea ninguna medida en contra de éste. Podría seguir el protocolo suponiendo 
    la desconexión del participante malicioso, lo que sería análogo al punto anterior, 
    pero tal vez existan medidas más drásticas como suspensión por un cierto tiempo 
    de participar en otras sesiones del protocolo, o derechamente la expulsión del 
    participante para siempre. 
    
    \item Optimizar recepción de mensajes: una parte importante a analizar por gente 
    más experta en el área de Redes es la manera en que se están recibiendo los 
    mensajes por parte de los participantes. Actualmente se delegó toda responsabilidad 
    a \emph{ZeroMQ}, el cual emplea una cola para no perder los mensajes entrantes 
    y la implementación actual se queda esperando entre mensajes, sin realizar 
    ninguna operación. Tal vez sea necesario revisar ese protocolo y optimizar la recepción de mensajes, abriendo múltiples \emph{sockets}, o utilizando el tiempo de espera entre mensajes para poder realizar alguna operación criptográfica pendiente, 
    y así no tener tiempo ocioso.
    
    \item Refinar protocolo criptográfico: 
    
    \item Mejorar diseño de aplicación móvil:
    
    \item Utilizar aplicación a través de Internet:
    
    \item Pruebas de seguridad:
\end{enumerate}