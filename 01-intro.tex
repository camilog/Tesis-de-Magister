\chapter{Introducción}\label{cap1}

\section{Anonimato en Internet}

Fruto tanto de las primeras revelaciones del grupo 
\emph{Wikileaks}\footnote{\url{https://wikileaks.org/}} en el año 2006, 
así como de la información revelada por el ex agente de la \emph{CIA}, 
\emph{Edward Snowden}\footnote{\url{http://www.huffingtonpost.com/news/nsa/}} 
en el año 2013, muchos usuarios han modificado su comportamiento de navegación 
por \emph{Internet} simplemente por el hecho de tener la presunción que todos 
sus movimientos están siendo registrados por organismos de Estado, a pesar de 
no ser amenaza a la seguridad de ninguna nación.

Por ende, usuarios de todo el mundo han comenzado a reclamar por su derecho a 
la privacidad\footnote{\url{http://www.livescience.com/37398-right-to-privacy.html}}, por ejemplo, tomando 
acciones que ocultan las páginas que visitan navegando en Internet de los 
organismos de Estado, o de cualquier tercero, que estén monitoreando y/o 
recopilando dichas visitas. Existe una basta comunidad compuesta por 
activistas de derechos humanos, abogados, ONGs, académicos y profesionales 
que abogan por que se haga valer el legítimo derecho a la 
privacidad de los usuarios de Internet, argumentando la importancia del 
anonimato\footnote{\url{https://www.derechosdigitales.org/anonimato/}} 
\footnote{\url{http://www.ted.com/talks/glenn_greenwald_why_privacy_matters}}, 
y motivando su tratamiento desde una perspectiva científica, brindando 
herramientas y protocolos que puedan asegurar el anonimato en Internet de 
manera segura, eficaz y eficiente. El trabajo de esta tesis está motivado por 
dicho objetivo.

Hoy en día navegar de manera anónima en Internet se realiza principalmente 
usando el software \emph{TOR}\footnote{\url{https://www.torproject.org/}}, 
el cual en su versión más popular consiste en un explorador web (variante de 
\emph{Firefox}\footnote{\url{https://www.mozilla.org/en-US/firefox}}) que 
ejecuta el protocolo \emph{onion-routing} (detallado en la siguiente sección), 
el cual permite dos objetivos: (1) ocultar la identidad del usuario tanto al 
proveedor del servicio que está consumiendo (página web que está visitando), 
como de algún observador externo, y (2) acceder a servicios ``ocultos'' que 
solo son accesibles por medio de \emph{TOR}. En este trabajo nos 
concentraremos en el primer objetivo solamente, detallando la manera en que 
\emph{TOR} logra ocultar la identidad del usuario del servicio. Es importante 
nombrar que \emph{TOR} no es la única manera de navegar de manera anónima en 
Internet: protocolos como \emph{mix-networks} y \emph{DC-Nets} (tema central 
en esta tesis) también logran ocultar la identidad del usuario.

El anonimato, más allá de su atractivo científico, es una problemática 
socialmente difícil de abordar. Esto se debe a que invoca prejuicios y dilemas 
éticos que, puestos en un contexto poco informado, pueden llegar a generar 
aprehensión e incluso rechazo. Parte de las aprehensiones se deben a la 
posibilidad de realizar acciones ilícitas bajo el paraguas del anonimato, lo 
cual puede impedir la persecución policial de delitos donde el criminal use 
herramientas que permitan anonimato efectivo. Entre estas acciones se pueden 
mencionar: denuncias falsas, compra/venta de drogas, distribución de material 
pornográfico infantil, fraude, \emph{bullying}, etc. ¿Deberían los científicos 
crear y promover herramientas que permitan esas acciones? ¿Valen la pena las 
ventajas del anonimato como para permitir la realización de estas acciones sin 
la posibilidad de perseguir a sus ejecutores? Estas son algunas preguntas que 
todo científico debería realizarse antes de generar un trabajo (independiente 
del tema de investigación) que afecta directamente la sociedad en que está 
envuelto. Desde el punto de vista del autor, la labor del científico es 
siempre generar el ``mejor conocimiento'' posible, expandir la barrera del 
conocimiento humano, y al mismo tiempo, estar consciente de las posibles 
implicancias que posea en otras esferas de la sociedad, como por ejemplo en 
este caso, facilitar acciones ilícitas a través de Internet. Pese a los 
potenciales usos negativos, es nuestra opinión que los sistemas de anonimato 
pueden ser herramientas efectivas contra la censura y violación de la 
privacidad existentes en regímenes autoritarios. También en situaciones de 
denuncia donde los potenciales informantes arriesgan su vida (denuncias de 
narcotráfico o espionaje). El problema de limitar el uso socialmente negativo 
de estas tecnologías es un problema de investigación abierto y extremadamente 
interesante.  

\section{Protocolos de Anonimato}

\subsection{\emph{Mix-networks}}

Una alternativa para anonimizar mensajes lo entregan protocolos de mezcla de 
mensajes, comúnmente llamados \emph{mixnets}. Una \emph{mixnet} consiste en un 
proceso iterativo donde un conjunto de mensajes encriptados es mezclado 
(permutado) en forma secuencial por una colección de servidores o 
``\emph{mix-servers}'' \cite{chaum1981untraceable}. Este proceso puede verse 
como una ``mezcla no invertible'' donde la \emph{mixnet} disocia datos 
referentes a la emisión del mensaje (identidad, tiempo del envío, etc.) con el 
mensaje mismo, impidiendo individualizar a un emisor con un mensaje en 
particular. El principal problema de este protocolo es que el anonimato se 
preserva gracias a la participación de \emph{mix-servers} honestos, y por ende 
requiere de una ``tercera parte confiable''. Para garantizar la privacidad de 
la mezcla (esto es, el anonimato), es que se utilizan varias etapas 
(servidores) de mezcla, pues basta que un servidor sea honesto, para que la 
permutación realmente permanezca oculta. Sin embargo, el depender de varios 
servidores de mezcla repercute en un proceso más ineficiente, resultando en 
una alta latencia al momento de enviar un mensaje. Importante es mencionar que 
la seguridad que, dado que utilizan típicamente encriptación de clave pública, 
se logra con una \emph{mixnet} es solamente computacional, esto es, 
eventualmente en un futuro donde sea posible adquirir un mayor poder de 
cómputo, el anonimato podría verse comprometido.

\subsection{\emph{Onion-Routing}}

El protocolo más utilizado hoy en día para proveer anonimato en la red es 
\emph{onion-routing} \cite{reed1998anonymous}, utilizado principalmente a 
través del software \emph{TOR}. Su popularidad viene dada principalmente por 
el carácter abierto que posee el proyecto. Esto le permite recibir 
contribuciones por parte de voluntarios todos los 
días\footnote{\url{https://www.torproject.org/getinvolved/volunteer.html.en}}, 
arreglando fallas que puedan existir, además de recibir permanentes análisis 
por parte de instituciones académicas que buscan la mejora, tanto del software 
como del protocolo subyacente. Si bien el sistema posee múltiples ventajas 
(llegando a ser recomendado por el mismo Snowden para ocultarse de las 
intromisiones de organismos de 
Estado\footnote{\url{https://www.inc.com/larry-kim/5-online-privacy-tips-from-edward-snowden.html}}), se sabe que posee 
varias falencias. Por ejemplo la vulnerabilidad descrita en 
\cite{syverson2001towards}, llegando incluso a poder romper el anonimato que 
éste provee si se posee la capacidad de monitorear toda la red (como la que 
poseen los distintos \emph{ISP} que proveen la conexión a Internet a los 
usuarios de la red). También es importante mencionar que el protocolo 
\emph{onion-routing} fue pensado bajo la amenaza de un adversario ``local'', 
es decir, uno que no puede monitorear toda la red. Este supuesto últimamente 
se ha ido debilitando, puesto que varias filtraciones muestran que, la 
\emph{NSA} en particular, efectivamente poseería la capacidad de monitorear 
todo Internet, haciendo inseguro al protocolo \emph{onion-routing}, y en 
consecuencia el software \emph{TOR}.

\section{La Cena de Criptógrafos}

David Chaum en el año 1985 propuso un protocolo que permitía el envío de 
mensajes de manera anónima entre un grupo de participantes 
\cite{Chaum:1985:SWI:4372.4373, chaum1988dining}. Dicho protocolo fue motivado 
con una historia llamada ``El Problema de la Cena de Criptógrafos'' 
(\emph{Dining Cryptographers Problem}), razón por la cual el protocolo desde 
entonces se conoce como \emph{DC-Net} (\emph{Dining Cryptographers Network}).

El problema es el siguiente: están tres criptógrafos cenando tranquilamente en 
su restorán favorito. Al terminar la cena, el mozo se acerca a su mesa y les 
comunica que su cena ya está pagada y que pueden irse a sus casas. Los 
criptógrafos, altamente consternados, se miran mutuamente y llegan a la 
conclusión de que debió haber sucedido una de dos cosas: (1) uno de ellos pagó 
la cuenta de manera anónima (haciéndose el amable e invitando al resto sin que 
ellos sepan), ó (2) la cuenta fue pagada por alguien distinto a ellos tres 
(como la \emph{NSA} por ejemplo), revelando la intromisión del organismo de 
inteligencia de EE.UU. en la vida de los criptógrafos. En esta situación, se 
hace necesario poder dilucidar este problema, pero sin comprometer (si es que 
fue el caso) al criptógrafo que pagó la cuenta de manera anónima. Por lo tanto 
los criptógrafos necesitan saber si uno de ellos pagó la cuenta (sin saber 
quién) o si fue alguien distinto.

David Chaum generaliza el problema de la siguiente manera: se tienen 3 
participantes (los criptógrafos) donde cada uno de ellos quiere comunicar un 
mensaje (pagó o no pagó la cena). El resultado del protocolo debe entregar, ya 
sea el mensaje de alguno de ellos (que pagó la cena), sin conocer el emisor de 
dicho mensaje, o bien que todos los mensajes fueron iguales (revelando que 
nadie pagó la cena, concluyendo que fue un agente externo). El protocolo debe 
poder mantener el anonimato del participante que envió el mensaje tanto para 
el resto de los participantes como para cualquier observador externo que esté 
monitoreando las conversaciones.

La solución propuesta supone el envío de solo un bit de información (si pagó o 
no pagó la cena). La solución generalizada consiste en los siguientes pasos:
\begin{enumerate}
    \item Cada par de participantes $(p_i, p_j)$ escogen un bit al azar 
    compartido $b_{ij}$. Además, se fija $b_{ij} = -b_{ji}$
    \item Cada participante $p_i$ calcula $b_i$ como el \texttt{XOR} binario 
    (denotado $\oplus$) entre todos los $b_{ij}$ que posee compartidos. Por 
    ejemplo, en el caso de la cena: $b_1 = b_{12} \oplus b_{13}$, $b_2 = b_{12}
     \oplus b_{23}$, $b_3 = b_{13} \oplus b_{23}$.
    \item Cada participante $p_i$ define su propio $M_i$, el cual 
    corresponderá al bit que quiere comunicar ($M_i = 1$ si pagó la cena, $0$ 
    si no).
    \item Cada $p_i$ revelará el valor $o_i = M_i \oplus b_i$.
    \item Cualquier $p_i$ u observador externo puede calcular el valor 
    $D = \displaystyle\bigoplus_i o_i$.
    \item Si $D = 1$, entonces uno de los participantes envió el mensaje 1 
    (alguien pagó la cena), si $D = 0$, todos los participantes enviaron el 
    mensaje 0 (la cena la pagó un agente externo a los participantes).
\end{enumerate}

\begin{figure}[H]
\centering
\begin{subfigure}[b]{0.4\textwidth}
    \begin{footnotesize}
\newcommand{\participK}[2]{
\begin{scope}[shift={(#1)}]
\draw [draw,fill=lightgray] (0pt,10pt) circle (5pt);
\draw [draw,fill=lightgray] (10pt,0pt) arc (0:180:10pt and 5pt);
\fill [lightgray] (-10pt,-10pt) rectangle (10pt,0pt);
\draw [draw] (10pt,0pt) -- (10pt,-10pt);
\draw [draw] (-10pt,0pt) -- (-10pt,-10pt);
\draw [draw] (5pt,-1pt) -- (5pt,-10pt);
\draw [draw] (-5pt,-1pt) -- (-5pt,-10pt);
\draw [anchor=center] (0pt,-2.5pt) node {#2};
\end{scope}
}
\begin{tikzpicture}
\begin{scope}
\path (18:6.9em) coordinate (P3);
\path (90:9em) coordinate (P2);
\path (162:6.9em) coordinate (P1);
\draw [thick] (P3) to node [anchor=south ,pos=0.5,swap,sloped] {$b_{23} = 0\ $} (P2);
\draw [thick] (P3) to node [anchor=south ,pos=0.5,swap,sloped] {$b_{13} = 1\ $} (P1);
\draw [thick] (P2) to node [anchor=south ,pos=0.5,swap,sloped] {$b_{12} = 1\ $} (P1);

\participK{P1}{$P_1$};
\participK{P2}{$P_2$};
\participK{P3}{$P_3$};
\end{scope}


\end{tikzpicture}
\end{footnotesize}
    \caption{Ilustración del bit al azar $b_{ij}$ compartido entre cada par de 
    participantes.}
    \label{2a}
\end{subfigure}
~
\begin{subfigure}[b]{0.4\textwidth}
    \begin{footnotesize}
\newcommand{\participK}[2]{
\begin{scope}[shift={(#1)}]
\draw [draw,fill=lightgray] (0pt,10pt) circle (5pt);
\draw [draw,fill=lightgray] (10pt,0pt) arc (0:180:10pt and 5pt);
\fill [lightgray] (-10pt,-10pt) rectangle (10pt,0pt);
\draw [draw] (10pt,0pt) -- (10pt,-10pt);
\draw [draw] (-10pt,0pt) -- (-10pt,-10pt);
\draw [draw] (5pt,-1pt) -- (5pt,-10pt);
\draw [draw] (-5pt,-1pt) -- (-5pt,-10pt);
\draw [anchor=center] (0pt,-2.5pt) node {#2};
\end{scope}
}
\begin{tikzpicture}

\begin{scope}
\path (0em,7em) coordinate (Q1);
\path (0em,3.5em) coordinate (Q2);
\path (0em,0em) coordinate (Q3);

\node [rectangle, right of=Q1, anchor=west, node distance=1.5em] {$o_1 = M_1 \oplus b_1 = 0 \oplus 1 \oplus 1 = 0$};
\node [rectangle, right of=Q2, anchor=west, node distance=1.5em, text width=136pt] {(pagó)\\$o_2= M_2 \oplus b_2 = 1 \oplus 1 \oplus 0 = 0$};
\node [rectangle, right of=Q3, anchor=west, node distance=1.5em] {$o_3= M_3 \oplus b_3 = 0 \oplus 1 \oplus 0 = 1$};

\participK{Q1}{$P_1$};
\participK{Q2}{$P_2$};
\participK{Q3}{$P_3$};
\end{scope}

\end{tikzpicture}
\end{footnotesize}

    \caption{Cálculo de $o_i$ por cada participante, dependiendo si pagó o no 
    la cena.}
    \label{2b}
\end{subfigure}

\protect\caption{Ejemplo de Cena de Criptógrafos donde uno de los 
participantes pagó la cena. En este caso $D = 0 \oplus 0 \oplus 1 = 1$. 
No es difícil observar que si ningún participante pagó la cena, $D = 0$.}
\label{fig:example_dcnet_chaum}
\end{figure}

Luego de proponer esta solución, David Chaum realiza el análisis de seguridad 
del protocolo (para detalles consultar el \emph{paper} original 
\cite{chaum1988dining}), del cual concluye que el protocolo sugerido entrega 
anonimato de manera \emph{incondicional}: que incluso un adversario que posea 
todo el poder de cómputo posible no puede dar con una estrategia que le 
permita encontrar al emisor del mensaje con mejor probabilidad que $1/n$ 
(donde $n$ es la cantidad total de participantes).

Posteriormente se propone una manera de generalizar el protocolo para 
utilizarlo con mensajes más largos (de más de 1 bit), lo cual será 
visto en la siguiente sección.

\section{\emph{DC-Net} como protocolo de anonimato}

\emph{DC-Net} (\emph{Dining Cryptographers Network}) es un protocolo que 
permite enviar un mensaje de manera anónima a un grupo de participantes 
(llamado \emph{anonymity set}). Este protocolo protege la identidad del emisor 
del mensaje de manera \emph{incondicional}, es decir, que ningún adversario 
(independiente del poder de cómputo que posea) puede encontrar a dicho emisor 
con una probabilidad mayor a $1/n$ (donde $n$ es el tamaño del 
\emph{anonymity set} o la cantidad total de participantes). Dicho adversario 
puede ser interno (uno de los participantes) o externo (por ejemplo, un 
observador externo que esté monitoreando todas las comunicaciones entre los 
distintos participantes).

A continuación se detallará la generalización del protocolo para permitir el 
envío de mensajes de largo arbitrario. Supongamos un \emph{anonymity set} de 
$n$ participantes $\{p_1, p_2, \ldots, p_n\}$. Sin pérdida de generalidad, 
supondremos que $p_1$ es el único participante que quiere enviar un mensaje 
(más adelante se discutirá el caso cuando dos o más participantes quieren 
enviar un mensaje). En este caso $M_1 = m \in 
\mathbb{Z}_q$\footnote{$\mathbb{Z}_q = \{1, 2, \ldots, q - 1\}$}, para todo el 
resto $M_i = 0$ $(\forall i \in \{2, \ldots, n\})$.

El primer paso, al igual que en el protocolo original, es generar valores (o 
claves) compartidos entre cada par de participantes. Cada par $\{p_i, p_j\}$ 
comparte un valor único $k_{ij}$ (y se define además que $k_{ij} = -k_{ji}$ y 
$k_{ii} = 0$).

\begin{figure}[H]
  \centering
    \includegraphics[width=0.5\textwidth]{imagenes/dcnet-general-03.png}
  \caption{Claves compartidas en DC-Net}
\end{figure}

Con esto, cada participante $p_i$ genera un valor igual a la suma de todas las 
claves compartidas que posee, esto es $K_i = \sum_{j=1}^n k_{ij}$. 
Luego, genera el valor $O_i = K_i + M_i$ que comunica al resto de los 
participantes. Este valor es la suma de la clave $K_i$ con el mensaje $M_i$, 
(recordemos que solo $p_1$ posee su mensaje distinto a 0, por lo que para el 
todo el resto de los participantes $O_i = K_i$). Finalmente, cada participante 
envía el valor $O_i$ al resto de los participantes (vía \emph{broadcast}, por 
ejemplo). Con esto, los valores $O_i$ quedan disponibles tanto para el resto 
de los participantes, como para cualquier observador externo.

Ahora solo queda encontrar el mensaje enviado por $p_1$. Para ello, se calcula 
el valor $D$ como la suma de todos los $O_i$ enviados por los participantes: 
$$D = \sum_{i=1}^n O_i \overset{(1)}{=} \sum_{i=1}^n K_i + \sum_{i=1}^n M_i 
\overset{(2)}{=} \sum_{i=1}^n K_i + m \overset{(3)}{=} m$$

La primera (1) igualdad se debe a la definición de $O_i$ y la separación de la 
suma en las dos componentes $K_i$ y $M_i$. La segunda (2) hace referencia al 
hecho que $M_i = 0$ $\forall i \geq 2$, por lo que solo ``sobrevive'' 
$M_1 = m$. Finalmente la tercera (3) se debe al hecho que las claves 
compartidas se cancelan mutuamente, debido a que $k_{ij} = -k_{ji}$. Por lo 
tanto, el valor $D$ corresponde al único mensaje enviado $m$, el cual es 
revelado al resto de los participantes, manteniendo en el anonimato a su 
emisor (en este caso, $p_1$).

Informalmente, el mensaje $m$ se oculta entre la suma de las claves 
compartidas. En principio, sin conocer el valor de todas las claves compartidas (a 
menos que se coludan $n-1$ participantes), no es claro como poder dilucidar si 
un mensaje $m$ vino de un cierto valor $O_i$ revelado por el participante $p_i$.

Ahora bien, el protocolo anteriormente descrito sufre de dos problemas: (1) el 
protocolo funciona sólo cuando un único participante envía un mensaje (de 
haber dos o más participantes con $M_i \neq 0$, se tendría que $D = \sum M_i$, 
por lo que sería imposible rescatar los mensajes individuales). Y, (2) no 
evita que participantes maliciosos envíen mensajes erróneos: por ejemplo, que 
participantes maliciosos envíen valores de claves compartidas distintos, 
causando que estas no se cancelen, evitando así la revelación del mensaje $m$ 
final. En este trabajo se detallará una variante que soluciona estos problema 
utilizando herramientas criptográficas que serán presentadas en el próximo 
capítulo.

\section{Objetivos del Trabajo}

\subsection{Objetivo General}

Diseñar e implementar una variante del protocolo \emph{DC-Net} que alcance 
mejores niveles de seguridad que los ofrecidos por otros protocolos 
actualmente utilizados para mantener anonimato. Además la variante debe estar 
optimizada para sistemas completamente distribuidos donde la implementación 
presente mejores resultados que las alternativas existentes, tomando en cuenta 
distintos factores como el tipo de seguridad alcanzada y la eficiencia en el 
envío de los mensajes.

\subsection{Objetivos Específicos}

\begin{itemize}
    \item Analizar las actuales soluciones (protocolos) que brindan anonimato 
    con el fin de evaluarlos críticamente y mejorarlos o contrastarlos con la 
    solución propuesta en el presente trabajo.
    \item Analizar otras implementaciones de variantes de \emph{DC-Net}, 
    identificando aspectos a mejorar, a fin de proponer una solución que 
    permita disminuir los supuestos de seguridad en que se basan dichas 
    soluciones.
    \item Testear la implementación realizada en distintos escenarios, 
    analizando el tiempo de ejecución en cada uno de ellos, con el objetivo 
    de encontrar el mejor contexto en que se desempeñaría la solución final.
    \item Crear un caso de prueba del sistema implementado, que sirva como 
    motivación a distintos posibles usos que se le pueda dar al código 
    realizado.
\end{itemize}

\section{Organización del Documento}

En el presente documento se describen los pasos que se siguieron para la 
propuesta de un diseño y una implementación de un sistema de mensajería que 
provee anonimato a los emisores basado en el protocolo \emph{DC-Net}. En el 
Capítulo \ref{cap2} se describen los antecedentes criptográficos 
necesarios para comprender la solución propuesta, además de un análisis de 
soluciones similares propuestas en otros trabajos. Luego en el Capítulo 
\ref{cap3} se describe de manera detallada el diseño de la variante propuesta, 
especificando paso a paso lo que cada uno de los participantes debe realizar 
para poder contar con el sistema que se propuso como objetivo. Posteriormente 
en el Capítulo \ref{cap4} se presentan detalles de la implementación del 
diseño descrito anteriormente, incluido decisiones de diseño y cómo se fueron 
tomando para dar forma al sistema finalmente implementado. En este capítulo 
también se entrega una breve descripción de un caso de uso, una aplicación 
móvil. En el Capítulo \ref{cap5} se detallan los experimentos que se llevaron 
a cabo para testear la solución, incluyendo la evaluación de los tiempos de 
ejecución bajo distintos escenarios. Los resultados son discutidos en la parte 
final del capítulo. Luego en el Capítulo \ref{cap6} se brindan posibles 
mejoras tanto al protocolo diseñado como a la implementación realizada. 
Finalmente en el Capítulo \ref{cap7} se presentan tanto las conclusiones del 
presente trabajo, como las experiencias que dejaron tanto su diseño como su 
implementación.